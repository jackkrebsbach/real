\documentclass{report}

\input{preamble}
\input{macros}
\input{letterfonts}
\newcommand{\contra}{
$\rightarrow\!\leftarrow$
}

\title{Ultimate Problem Set}
\author{Jack Krebsbach }

\date{Dec 11th}

\begin{document}
\maketitle



\qs{}{


Suppose that $x>-1$ and that $x \neq 0$. Prove that
$$
(1+x)^n>1+n x
$$
for each integer $n>1$. This result is know as Bernoulli's inequality.
}

\begin{myproof}
We will show that this inequality holds for $x>-1$ and $x\not= 0$ by induction. First, we see when $n=2$ that $$ (1+x)^2 = 1 +2x + x^2 > 1+2x.$$ Thus, our base case holds. Now we assume that $$ (1+x)^n > 1 +nx$$ is true. We want to show that
$ (1+x)^{n+1} > 1 +(n+1)x$
is also true. First, we multiply both sides by $(1+x)$, then $$(1+x)(1+x)^n > (1+x)(1+nx)$$
$$ \implies (1+x)^{n+1} > 1 + x + nx + nx^2 \geq 1 + nx + x = 1 + (n+1)x.$$

Hence, $(1+x)^n>1+n x$ for each integer $n>1$.

\end{myproof}

\pagebreak
\qs{}{

Show that $e$ is irrational by supposing that $e=\frac{m}{n}$ and deriving a contradiction. Use the fact that $e=\sum_{j=0}^{\infty} \frac{1}{j !}$. Let $s_k=\sum_{j=0}^k \frac{1}{j !}$.
}

(a) Prove that

$$
e-s_k<\frac{1}{(k+1) !}\left\{1+\frac{1}{k+1}+\left(\frac{1}{k+1}\right)^2+\cdots\right\} .
$$

\begin{myproof}
    
We have that
$$ e-s_k = \sum_{j=0}^{\infty} \frac{1}{j !} - \sum_{j=0}^k \frac{1}{j !} = \sum_{j=k+1}^{\infty} \frac{1}{j !} = \frac{1}{(k+1)!} +\frac{1}{(k+2)!}+ \frac{1}{(k+3)!} + \cdots$$ $$= \frac{1}{(k+1)!}\left[1 + \frac{1}{(k+2)}  + \frac{1}{(k+2)(k+3)} + \cdots \right]$$
$$< \frac{1}{(k+1)!}\left[1 + \frac{1}{(k+1)}  + \frac{1}{(k+1)(k+1)} + \cdots \right]$$
$$= \frac{1}{(k+1)!}\left[1 + \frac{1}{(k+1)}  + \frac{1}{(k+1)^2} + \cdots \right]$$
\bigskip
\end{myproof}
\bigskip
(b) Prove that $e-s_k<\frac{1}{k(k!)}$ for all $k \in \mathbb{N}$.

\begin{myproof}
  Let $$ y_n = \sum_{n=0}^{m} \frac{1}{(k+1)^n} .$$ Then $$y_n - \frac{1}{(k+1)^n} y_n = \sum_{n=0}^{m} \frac{1}{(k+1)^n} - \sum_{n=1}^{m+1} \frac{1}{(k+1)^n} $$ 
  $$\implies y_n\left(1 - \frac{1}{k+1}\right) = \frac{1}{k+1} - \frac{1}{(k+1)^{m+1}} $$   

$$\implies y_n = \frac{\frac{1}{k+1} - \frac{1}{(k+1)^{m+1}}}{\left(1 - \frac{1}{k+1}\right)}.$$


Now let $$ \lim_{m \rightarrow \infty} y_n = \lim_{m \rightarrow \infty}\frac{\frac{1}{k+1} - \frac{1}{(k+1)^{m+1}}}{\left(1 - \frac{1}{k+1}\right)}$$
$$ \implies y_n = \frac{1}{k+1}\frac{1}{\left(1 - \frac{1}{k+1}\right)}$$
$$  = \frac{1}{k+1}\frac{k+1}{k+1-1}$$
$$  = \frac{1}{k+1}\frac{k+1}{k} = \frac{1}{k}.$$


Now, $$ e - s_k  < \frac{1}{(k+1)!}\left[1 + \frac{1}{(k+1)}  + \frac{1}{(k+1)^2} + \cdots \right] = \frac{1}{(k+1)!}\{y_k\} \leq \frac{1}{(k+1)!} \frac{1}{k} < \frac{1}{k(k!)}$$
\end{myproof}

\bigskip

(c) If $e=\frac{m}{n}$, prove that $n ! e$ and $n ! s_n$ are integers.

\begin{myproof}

We have $$ n!e = n!\frac{m}{n} = (n-1)!m.$$ Since the integers are closed then $n!e$ must be an integer. 
    
\end{myproof}

\begin{myproof}

$$ n!s_n = n!s_k=\sum_{j=0}^k \frac{1}{j !}(n-1)! = n!\left(1 + \frac{1}{2!} + \frac{1}{3!}+ \cdots +\frac{1}{n!}\right) .$$
$$ = n! + \frac{n!}{2!} + \frac{n!}{3!}+ \cdots + 1 .$$

$$ = n! + n(n-1)(n-2)\cdots(4)(3) + n(n-1)(n-2)\cdots(5)(4)+ \cdots + 1 .$$

Again, because the integers are closed  it must be that $n!s_n$ is also an integer.
\end{myproof}

\bigskip
(d) If $e=\frac{m}{n}$, prove that $n!\left(e-s_n\right)$ is an integer between 0 and 1 , which is absurd.

\bigskip
\begin{myproof}

Consider that $$n!\left(e-s_n\right) < n!\frac{1}{n!n}= 1/n,$$

which means that $0< n!\left(e-s_n\right) <1.$  Because  $n!(e-s_n)$ is an integer we have encountered a contradiction, this is impossible. Thus, $e$ can not be a rational number.
    
\end{myproof}
\bigskip
\pagebreak
\qs{}{
Let $f$ be a function defined on all of $\mathbb{R}$, and assume there is a constant $c$ such that $0<c<1$ and
$$
|f(x)-f(y)| \leq c|x-y|
$$
for all $x, y \in \mathbb{R}$.
}

(a) Show that $f$ is continuous.
\par
\bigskip
\begin{myproof}
    
Let $\epsilon > 0$ and choose $\delta= \epsilon/c$. Now if we have $$|x-y| < \delta$$ then $$|f(x) - f(y)| \leq c|x-y| < c \frac{\epsilon}{c} = \epsilon.$$
Thus, $f$ must be continuous.
\end{myproof}

\bigskip
\bigskip
(b) Pick some $y_1 \in \mathbb{R}$ and construct the sequence
$$
\left(y_1, f\left(y_1\right), f\left(f\left(y_1\right)\right), \ldots\right) .
$$

In general, if $y_{n+1}=f\left(y_n\right)$, show that the resulting sequence $\left(y_n\right)$ is a Cauchy sequence. Hence we may let $y=\lim _{n \rightarrow \infty} y_n$.

\bigskip
\begin{myproof}
  Let $\epsilon > 0.$ First we will show that $$|y_{n+1} - y_n| \leq c^{n-1}|y_2 - y_1|=c^{n}\frac{|y_2 - y_1|}{c}$$ is true by induction:
\bigskip
\par
\begin{enumerate}
  \item Base case: From the given, when $n=2$, $$|y_3 - y_2| =|f(y_2) - f(y_1)| \leq c^1|y_2 -y_1|.$$
  \item Inductive step: Now we want to show that if this holds true for $n$, this also holds true for $n+1$. We assume  $$ |y_{n+1} - y_n| \leq c^{n-1}|y_2 - y_1|$$ is true. We have $$ c|y_{n+1} - y_n| \leq c^{n}|y_2 - y_1|.$$ It follows
    $$|y_{n+2} - y_{n+1}| =|f(y_{n+1}) - f(y_{n})| \leq c|y_{n+1} - y_n| \leq c^{n}|y_2 - y_1|.$$
\end{enumerate}


Finally, we conclude that  $$|y_{n+1} - y_n| \leq c^{n-1}|y_2 - y_1|=c^{n}\frac{|y_2 - y_1|}{c}$$
for all $n > 1.$

\pagebreak
\bigskip
\par
Now we may continue in the proof showing that $(y_n)$ is Cauchy. Let $$B = \frac{|y_2 - y_1|}{c}.$$
  Now, choose $N \in \NN$ such that for all $n>N$,$$\frac{c^n}{(1/B  - c^{n})}< \epsilon.$$ Let  $m > n > N.$  We have that  
  $$ |x_m - x_n|= |x_m - x_{m-1} + x_{m-1} - x_{m-2} + x_{m-2} + \dots + x_{n+1}- x_n| $$
$$\leq |x_m - x_{m-1}| + |x_{m-1} - x_{m-2}| +  \dots + |x_{n+1}- x_n| $$

$$= \sum_{n}^{m-1}|x_{n+1} - x_n| $$

$$\leq \sum_{n}^{m-1}c^{n}B.$$

If $$y_n =\sum_{n}^{m-1} c^{n},$$ then $$y_n  - c^{n}By_n = \sum_{n}^{m-1} c^{n}B - \sum_{n+1}^{m} c^{n}B$$
$$\implies y_n(1  - c^{n}B) = Bc^n - c^mB $$

$$\implies y_n = \frac{B(c^n - c^m) }{(1  - c^{n}B)}= \frac{c^n - c^m }{(1/B  - c^{n})} $$
$$< \frac{c^n}{(1/B  - c^{n})} < \epsilon.$$
Hence, $(y_n)$ is Cauchy and we may let $y=\lim _{n \rightarrow \infty} y_n$
\end{myproof}

\par
\bigskip
(c) Prove that $y$ is a fixed point of $f$ (i.e. $f(y)=y$ ) and that it is unique in this regard.

\par

\begin{myproof}
    
\bigskip
Consider that $y_{n+1} = f(y_n).$ Then, $$ \lim_{n \rightarrow \infty} y_{n+1} = \lim_{n \rightarrow \infty}f(y_n)$$
$$\implies y=  f\left(\lim_{n \rightarrow \infty}y_n\right)$$

$$\implies  y = f(y).$$

Thus, $y$ is a fixed point. Consider, by way of contradiction, that $y$ is not the only fixed point and there exists another fixed point $x$ where $y\not=x.$

Then, $$|f(y) - f(x)| = |x-y| < c|x-y|$$ which is a contradiction since $0 < c < 1$. Thus, it must be that $y$ is a unique fixed point.

\end{myproof}
\bigskip
(d) Finally, prove that if $x$ is any arbitrary point in $\mathbb{R}$, then the sequence $(x, f(x), f(f(x)), \ldots)$ converges to $y$ (as defined in (b)).

\begin{myproof}
Because we proved previously that for any arbitrary $x \in \RR$ the sequence $(y_n) \rightarrow y$ and $y$ is a unique fixed point it must be the sequence $(x, f(x), f(f(x)), \ldots)$ converges to $y$ (as defined in (b)).
\end{myproof}

\pagebreak
\qs{}{
Let $\left\{r_n\right\}$ be a listing of all the rational numbers. Define a function $f$ by $f(x)=0$ if $x$ is irrational and $f\left(r_n\right)=1 / n$ for all $n$. Show that $f$ is continuous everywhere except for the set of rational numbers.
}

\begin{myproof}

  Let $\epsilon > 0.$ We consider if $f$ is continuous at $c$, an irrational number (1) and if is continuous at $c$, a rational number (2). Note that $f(x) \geq0 $ for all $x \in \RR.$

  \begin{enumerate}
    \item 
  By the Archimedes Principle there exists $N \in \NN$ such that for all $n>N,$ $1/n < \epsilon.$ Consider all the mappings of rational numbers where $n\leq N$, and choose $\delta = \min\{|r_n - c|\}/2$.  Note that we have chosen $N\in\NN$ such that for all rational numbers where $n>N$, $f(r_n)< \epsilon$ and also for all $x \in \RR \setminus \QQ$, $f(x)=0< \epsilon.$ Thus, when $x\in \RR$ we have chosen $\delta$ such that when $|x-c|< \delta$, we automatically have that $|f(x) - f(c)|= |f(x) - 0| = |f(x)| = f(x)<\epsilon.$ Hence, $f$ is continuous on the irrationals.
\item Consider, by way of contradiction, that $f$ is continuous on the rational numbers. Then there exists $\delta$ such that when $x\in \RR$ and $|x-y| <\delta$ we automatically have $|f(x) - f(c)| < f(c)/2.$ By the density of the irrational numbers in $\RR$ there exists an irrational number $,x_I,$ such that $|x_I - y| < \delta.$ It follows $|f(x_I) - f(c)|= |0 - f(c)| = f(c) < f(c)/2 \implies 1 < 1/2,$ a contradiction. Thus, $f$ can not be continuous on the rationals.
      
  \end{enumerate}

    
\end{myproof}
\qs{}{

Using the $\delta-\epsilon$ definition of a limit, show
$$
\lim _{x \rightarrow 1} \frac{x^3-1}{x-1}=3 .
$$
}

\begin{myproof}
  Consider,
  $$ \lim _{x \rightarrow 1} \frac{x^3-1}{x-1}= \lim _{x \rightarrow 1}\left[ (x-1)\frac{x^2+x+1}{x-1}\right]= \lim _{x \rightarrow 1}x^2+x+1 = 3.$$ Let $\epsilon >0$ and choose $\delta = \min\{1,\epsilon/4\}$. Note, if restrict $\delta$ to be a maximum of $1$ then $|x+2| \leq |x| + 2 | \leq |2| + 2 = 4.$
 If we have $0<|x-1| < \delta$ then $$|f(x) - L|= |x^2 +x +1 - 3| = |x^2 +x- 2|= |(x+2)(x-1)| = |x+2||x-1|.$$


Altogether, $$|f(x) - L| = |x+2||x-1|<4  \frac{\epsilon}{4}= \epsilon.$$

Hence, 
$$\lim _{x \rightarrow 1} \frac{x^3-1}{x-1}=3.$$

\end{myproof}


\end{document}
