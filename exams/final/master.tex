\documentclass{report}

\input{preamble}
\input{macros}
\input{letterfonts}
\newcommand{\contra}{
$\rightarrow\!\leftarrow$
}

\title{Ultimate Problem Set}
\author{Jack Krebsbach }

\date{Dec 11th}

\begin{document}
\maketitle



\qs{}{


Suppose that $x>-1$ and that $x \neq 0$. Prove that
$$
(1+x)^n>1+n x
$$
for each integer $n>1$. This result is know as Bernoulli's inequality.
}

\begin{myproof}
We will show that this inequality holds for $x>-1$ and $x\not= 0$ by induction. First, we see when $n=2$ that $$ (1+x)^2 = 1 +2x + x^2 > 1+2x.$$ Thus, our base case holds. Now we assume that $$ (1+x)^n > 1 +nx$$ is true. We want to show that
$ (1+x)^{n+1} > 1 +(n+1)x$
is also true. We have that $$(1+x)(1+x)^n > (1+nx)(1+x)$$
$$ \implies (1+x)^{n+1} > 1 + x + nx + nx^2 \geq 1 + nx + x = 1 + (n+1)x.$$

\end{myproof}

\pagebreak
\qs{}{

Show that $e$ is irrational by supposing that $e=\frac{m}{n}$ and deriving a contradiction. Use the fact that $e=\sum_{j=0}^{\infty} \frac{1}{j !}$. Let $s_k=\sum_{j=0}^k \frac{1}{j !}$.
}

(a) Prove that

$$
e-s_k<\frac{1}{(k+1) !}\left\{1+\frac{1}{k+1}+\left(\frac{1}{k+1}\right)^2+\cdots\right\} .
$$

$$ e-s_k = \sum_{j=0}^{\infty} \frac{1}{j !} - \sum_{j=0}^k \frac{1}{j !} = \sum_{j=k+1}^{\infty} \frac{1}{j !} = \frac{1}{(k+1)!} +\frac{1}{(k+2)!}+ \frac{1}{(k+3)!} + \cdots$$ $$= \frac{1}{(k+1)!}\left[1 + \frac{1}{(k+2)}  + \frac{1}{(k+2)(k+3)} + \cdots \right]$$
$$< \frac{1}{(k+1)!}\left[1 + \frac{1}{(k+1)}  + \frac{1}{(k+1)^2} + \cdots \right]$$
\bigskip

\bigskip
(b) Prove that $e-s_k<\frac{1}{k(k!)}$ for all $k \in \mathbb{N}$.

\begin{myproof}
  Let $$ y_n = \sum_{n=0}^{m} \frac{1}{(k+1)^n} $$. Then $$y_n - \frac{1}{(k+1)^n} y_n = \sum_{n=0}^{m} \frac{1}{(k+1)^n} - \sum_{n=1}^{m+1} \frac{1}{(k+1)^n} $$ 
  $$\implies y_n\left(1 - \frac{1}{k+1}\right) = \frac{1}{k+1} - \frac{1}{(k+1)^{m+1}} $$   
  $$\implies y_n\left(1 - \frac{1}{k+1}\right) = \frac{1}{k+1} - \frac{1}{(k+1)^{m+1}} $$

$$\implies y_n = \frac{\frac{1}{k+1} - \frac{1}{(k+1)^{m+1}}}{\left(1 - \frac{1}{k+1}\right)}.$$


Now let $$ \lim_{m \rightarrow \infty} y_n = \lim_{m \rightarrow \infty}\frac{\frac{1}{k+1} - \frac{1}{(k+1)^{m+1}}}{\left(1 - \frac{1}{k+1}\right)}$$
$$ \implies \lim_{m \rightarrow \infty} y_n = \frac{1}{k+1}\frac{1}{\left(1 - \frac{1}{k+1}\right)}$$
$$  = \frac{1}{k+1}\frac{k+1}{k+1-1}$$
$$  = \frac{1}{k+1}\frac{k+1}{k} = 1.$$

\end{myproof}

\bigskip

(c) If $e=\frac{m}{n}$, prove that $n ! e$ and $n ! s_n$ are integers.

\begin{myproof}

We have $$ n!e = n!\frac{m}{n} = (n-1)!m.$$ Since the integers are closed under multiplication then $n!e$ must be an integer. 
    
\end{myproof}

\begin{myproof}

$$ n!s_n = n!s_k=\sum_{j=0}^k \frac{1}{j !}(n-1)! = n!\left(1 + \frac{1}{2!} + \frac{1}{3!}+ \cdots +\frac{1}{n!}\right) .$$
$$ = n! + \frac{n!}{2!} + \frac{n!}{3!}+ \cdots + 1 .$$

This is just a sum of integers, so it must be that $n!s_n$ is also an integer.
\end{myproof}

\bigskip
(d) If $e=\frac{m}{n}$, prove that $n!\left(e-s_n\right)$ is an integer between 0 and 1 , which is absurd.

\bigskip
\begin{myproof}

Consider that $$n!\left(e-s_n\right) < n!\frac{1}{n!n}= 1/n$$

Since $n\in\NN$ and $e-s_n$ is an integer we have encountered a contradiction, $1/n < 1.$ Thus, $e$ can not be a rational number.
    
\end{myproof}
\bigskip

\qs{}{
Let $f$ be a function defined on all of $\mathbb{R}$, and assume there is a constant $c$ such that $0<c<1$ and
$$
|f(x)-f(y)| \leq c|x-y|
$$
for all $x, y \in \mathbb{R}$.
}

(a) Show that $f$ is continuous.

\par

\bigskip
\bigskip
(b) Pick some $y_1 \in \mathbb{R}$ and construct the sequence
$$
\left(y_1, f\left(y_1\right), f\left(f\left(y_1\right)\right), \ldots\right) .
$$

In general, if $y_{n+1}=f\left(y_n\right)$, show that the resulting sequence $\left(y_n\right)$ is a Cauchy sequence. Hence we may let $y=\lim _{n \rightarrow \infty} y_n$.



\bigskip

\par
\bigskip
(c) Prove that $y$ is a fixed point of $f$ (i.e. $f(y)=y$ ) and that it is unique in this regard.

\par
\par
\bigskip
\bigskip
(d) Finally, prove that if $x$ is any arbitrary point in $\mathbb{R}$, then the sequence $(x, f(x), f(f(x)), \ldots)$ converges to $y$ (as defined in (b)).


\qs{}{
Let $\left\{r_n\right\}$ be a listing of all the rational numbers. Define a function $f$ by $f(x)=0$ if $x$ is irrational and $f\left(r_n\right)=1 / n$ for all $n$. Show that $f$ is continuous everywhere except for the set of rational numbers.
}

\qs{}{

Using the $\delta-\epsilon$ definition of a limit, show
$$
\lim _{x \rightarrow 1} \frac{x^3-1}{x-1}=3 .
$$
}





\end{document}
