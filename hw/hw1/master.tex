\documentclass{report}

\input{preamble}
\input{macros}
\input{letterfonts}

\title{HW \#1}
\author{Jack Krebsbach }

\date{Sep 6th}

\begin{document}

\maketitle

\qs{}{Let $n$ be a positive integer that is not a perfect square. Prove that $\sqrt{n}$ is irrational.}
\sol Assume, for contradiction, that $\sqrt{n}$ is a rational. Then $\sqrt{n}$ can be written in the form $\frac{a}{b}$ where $a,b \in \mathbb{Z}$ and $a,b$ are cooprime, or have no common factors. 

We have 


\begin{equation}
  \frac{a}{b} = \sqrt{n} \implies
  \left(\frac{a}{b}\right)^{2} = n \implies a^2 = n b^2
\end{equation}


This means that $n$ divides $a^{2}$. 
By the Fundemental Theorem of Arithmetic we can write $a$, $n$, and $b$ as  unique product of primes. 

Thus, 

\begin{equation}
 a^2 = n b^2 \implies
 \left(\prod^{k}_{i = 1} P_{i}^{n_i}\right)^2  = \prod_{j = 1}^{l} P_{j}^{m_j}\left(\prod^{t}_{k = 1} P_{k}^{l_k}\right)^2 
\end{equation}
After simplification of (2) we have
\begin{equation}
     \prod^{k}_{i = 1} P_{i}^{2n_i}  = \prod_{j = 1}^{l} P_{j}^{m_j}\prod^{t}_{k = 1} P_{k}^{2l_k}
\end{equation}


In both expressions of $a^2$ and $b^2$, as a product of primes, we have an even number of each prime in the product. Because $n$ is not a perfect square, there must be at least 1 prime that is expressed an odd number of times. We are then guaranteed that by expressing $nb^2$ as a product of primes there must be at least $1$ prime which appears an odd number of times. However, the left hand side of (3) clearly shows this is not the case $\rightarrow\!\leftarrow.$

With this contradiction we have no choice but to overturn our assumption and conclude that $\sqrt{n} \in \mathbb{R} \setminus \mathbb{Q} \qed$

\bigskip
\qs{}{
   Use the Principle of Mathematical Induction to prove:
$$
1^{2}+3^{2}+\cdots+(2 n-1)^{2}=\frac{4 n^{3}-n}{3} \quad \forall n \in \mathbb{N} $$  
}
\sol

Let $n=1 \in \mathbb{N}$. Then $1^2 =  \frac{4 (1)^{3}-1}{3} = 1$, showing that the equality holds for $n=1$. We asssume that 

$$ 1^{2}+3^{2}+\cdots+(2 n-1)^{2}=\frac{4 n^{3}-n}{3}, $$
is true and we procceed with induction on $n$. We want to show $P(n+1) = \frac{4 (n+1)^{3}-(n+1)}{3}$.

Consider
$$ 
1^{2}+3^{2}+\cdots+(2 n-1)^{2} +(2 (n+1)-1)^{2}=\frac{4 n^{3}-n}{3} + (2 (n+1)-1)^{2} 
$$  
$$ 
=\frac{4 n^{3}-n}{3} + (4n^2+4n +1) 
$$  

$$ 
=\frac{4 n^{3}-n+12n^2+12n +3}{3}
$$  

$$ 
=\frac{4 n^{3} + 8n^2 + 4n + 4n^2 + 8n + 4 -n -1}{3}
$$  
$$ 
=\frac{4 [n^{3} + 2n^2 + n + n^2 + 2n + 1] -(n +1)}{3}
$$  
$$ 
=\frac{4 [(n^{2} + 2n + 1)(n+1)] -(n+1)}{3}
$$  

$$ 
=\frac{4 (n+1)^3 -(n+1)}{3}.
$$  

Thus, $P(n+1) = \frac{4 (n+1)^{3}-(n+1)}{3}$, proving $1^{2}+3^{2}+\cdots+(2 n-1)^{2}=\frac{4 n^{3}-n}{3} \quad \forall n \in \mathbb{N} \qed$  

\bigskip
\qs{}{
 Let $n>1$ be a positive integer and let $a_{1}, a_{2}, \ldots, a_{n}$ be real numbers. Prove that

$$
\left|\sum_{k=1}^{n} a_{k}\right| \leq \sum_{k=1}^{n}\left|a_{k}\right|$$
}

\sol
Let $n=2 \in \mathbb{Z}^{+}$. Then 
$$
\left|\sum_{k=1}^{2} a_{k}\right| \leq \sum_{k=1}^{2}\left|a_{k}\right| \implies
\left|a_1 + a_2\right| \leq \left| a_1\right| + \left| a_2\right|,$$ which we know is true by the Triangle Inequality Theorem. We then want to show that $\left|\sum_{k=1}^{n+1} a_{k}\right| \leq \sum_{k=1}^{n+1}\left|a_{k}\right|$
.
We assume for proof by induction that

$$
\left|\sum_{k=1}^{n} a_{k}\right| \leq \sum_{k=1}^{n}\left|a_{k}\right|$$ is true.
Expanding yields
$$                                                                       
\left|a_1 + a_2 + \cdots + a_n\right| \leq \left|a_{1}\right| + \left|a_{2}\right|
+ \cdots +  \left|a_{n}\right|.$$

Adding $\left|a_{n+1}\right|$ to both sides results in
$$
\left|a_1 + a_2 + \cdots + a_n\right| + \left|a_{n+1}\right|  \leq \left|a_{1}\right| + \left|a_{2}\right|
+ \cdots +  \left|a_{n}\right|  + \left|a_{n+1}\right|,$$

and by the Triangle Inequality Theorem we have

$$
\left|a_1 + a_2 + \cdots + a_n + a_{n+1}\right|  \leq
\left|a_1 + a_2 + \cdots + a_n\right| + \left|a_{n+1}\right|  \leq \left|a_{1}\right| + \left|a_{2}\right|
+ \cdots +  \left|a_{n}\right|  + \left|a_{n+1}\right|.$$


Therefore,

$$
\left|\sum_{k=1}^{n+1} a_{k}\right| \leq \sum_{k=1}^{n+1}\left|a_{k}\right| $$  and we have indeed shown

$$
\left|\sum_{k=1}^{n} a_{k}\right| \leq \sum_{k=1}^{n}\left|a_{k}\right| \qed$$

\bigskip
\qs{}{
Exercise 1.2.8
}
(a) $f: \mathbb{N} \rightarrow \mathbb{N}$ where $f$ is 1-1 but not onto. 
\sol $f(x) = x^2$

(b) $f: \mathbb{N} \rightarrow \mathbb{N}$ where $f$ is not 1-1 but onto. 
\sol If $f$ is onto then for every $y \in \mathbb{N}$ there exists $x \in \mathbb{N}$ where $f(x) = y$. If we assume $f$  is \textit{not} 1-1 then we know that there exists $x_1 \neq x_2$ such that $f(x_1) = f(x_2)$. As $f$ is a function, we can not map $x \in \mathbb{N}$ to more than one $y \in \mathbb{N}$, then by pigenhole principle we must be missing at least one $y \in \mathbb{N}$. Therefore, $f$ can not be onto, which means this is impossible.

(c) $f: \mathbb{N} \rightarrow \mathbb{N}$ where $f$ is 1-1 and onto. 

\sol  
\[ 
 f(n) = 
  \begin{cases} 
  even & \frac{n}{2} (-1)^{n+1} \\
  odd & \frac{n-1}{2} (-1)^{n+1}
   \end{cases}
\]
\bigskip
\qs{}
{Exercise 1.2.12}

\qs{}
{Grade the following proof: Prove that $\sqrt{3}$ is irrational. Does a similar argument work to show $\sqrt{6}$ is irrational?}

Proof:To prove that $\sqrt{3}$ is irrational, we can use a proof by contradiction. The idea is to assume that $\sqrt{3}$ is rational, and then derive a contradiction.

Assumption: $\sqrt{3}$ is rational.

If $\sqrt{3}$ is rational, it can be expressed as a fraction $\frac{a}{b}$ in its simplest form, where $a$ and $b$ are integers with no common factors other than 1 (i.e., they are coprime), and $b$ is not equal to 0.

So, we have: $\sqrt{3}=\frac{a}{b}$

Squaring both sides of the equation: $3=\frac{a^{2}}{b^{2}}$

Rearranging the equation: $a^{2}=3 \cdot b^{2}$

From this equation, we can see that $a^{2}$ is a multiple of 3 , which implies that $a$ must also be a multiple of 3. Let's represent $a$ as $3 k$ where $k$ is an integer:

$$
\begin{gathered}
(3 k)^{2}=3 \cdot b^{2} \\
9 k^{2}=3 \cdot b^{2} \\
3 k^{2}=b^{2}
\end{gathered}
$$

Now, $b^{2}$ is also a multiple of 3 , which means $b$ must be a multiple of 3 . However, this contradicts our initial assumption that $a$ and $b$ have no common factors other than 1 (i.e., they are coprime). This contradiction arises from assuming that $\sqrt{3}$ is rational.

Hence, our initial assumption that $\sqrt{3}$ is rational must be false, which means that $\sqrt{3}$ is irrational. Now, to address whether a similar argument works for $\sqrt{6}$ :

No, a similar argument does not work for $\sqrt{6}$. The proof for the irrationality of $\sqrt{3}$ relies on the fact that it leads to a contradiction with respect to the common factors of $a$ and $b$. However, for $\sqrt{6}$, this argument does not apply directly because the factors of 6 are not the same as the factors of 3. In fact, $\sqrt{6}$ is not irrational; it is a rational number. It can be expressed as the fraction $\frac{2 \sqrt{3}}{3}$, which is a ratio of two integers.


\end{document}

