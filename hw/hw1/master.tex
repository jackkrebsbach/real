\documentclass{report}

\input{preamble}
\input{macros}
\input{letterfonts}

\title{HW \#1}
\author{Jack Krebsbach }

\date{Sep 6th}

\begin{document}

\maketitle

\qs{}{Let $n$ be a positive integer that is not a perfect square. Prove that $\sqrt{n}$ is irrational.}
\sol Assume, for contradiction, that $\sqrt{n}$ is a rational. Then $\sqrt{n}$ can be written in the form $\frac{a}{b}$ where $a,b \in \mathbb{Z}$ and $a,b$ are cooprime, or have no common factors. 

We have 


\begin{equation}
  \frac{a}{b} = \sqrt{n} \implies
  \left(\frac{a}{b}\right)^{2} = n \implies a^2 = n b^2
\end{equation}


Clearly $n$ divides $a^{2}$. 
By the Fundemental Theorem of arithmetic we can write $a$ and $n$ as a product of primes.

Thus, 
\begin{equation}
  \frac{a^2}{n} = \frac{\left(\prod^{k}_{i = 1} P_{i}^{n_i}\right)^2}{\prod_{j = 1}^{k} P_{j}^{m_j}} =\frac{
\left(\prod^{k}_{i = 1} P_{i}^{n_i}\right)
\left(\prod^{k}_{i = 1} P_{i}^{n_i}\right)
  }{\prod_{j = 1}^{k} P_{j}^{m_j}}  = b^2
\end{equation}.


Because $n$ divides $a^2$ we can re-write $b^2$ as the product


\begin{equation}
  n \left(\prod^{t}_{l = 1} P_{l}^{m_l}\right) = b^2
\end{equation}.


Clearly, $b^2 \geq n$, and it follows that $a \geq n$. Therefore we can rearrange $(2)$ yielding

\begin{equation}
  a^2 = (n)(a)\left(\prod^{z}_{i = 1} P_{i}^{m_i}\right)  \implies a = n\left(\prod^{z}_{i = 1} P_{i}^{m_i}\right) 
\end{equation}. 

Thus, $n$ divides $a$ in addition to $a^2$. Because of this we know that we can rewrite $a$ in terms of $n$, or $a = t(n)$ where $t \in \mathbb{Z}$. Then

\begin{equation}
  (tn)^2 = nb^2 \implies t^2n^2 = nb^2 \implies nt^2 = b^2 
\end{equation}

which means $n$ is a common factor of $b^2$, and by the preceding logic $b$ in addition $a$. Thus $a$ and $b$ can not be be cooprime  $\rightarrow\!\leftarrow$. This contradicts our initial assumption and we have no choice but to conclude that $\sqrt{n} \in \mathbb{R} \setminus \mathbb{Q} \qed$

\bigskip
\qs{}{
   Use the Principle of Mathematical Induction to prove:
$$
1^{2}+3^{2}+\cdots+(2 n-1)^{2}=\frac{4 n^{3}-n}{3} \quad \forall n \in \mathbb{N} $$  
}

\qs{}{
 Let $n>1$ be a positive integer and let $a_{1}, a_{2}, \ldots, a_{n}$ be real numbers. Prove that

$$
\left|\sum_{k=1}^{n} a_{k}\right| \leq \sum_{k=1}^{n}\left|a_{k}\right|$$
}


$$
\left|\sum_{k=1}^{n} a_{k}\right| \leq \sum_{k=1}^{n}\left|a_{k}\right|
$$
\qs{}{
Exercise 1.2.8
}

\qs{}
{Exercise 1.2.12}

\qs{}
{Grade the following proof: Prove that $\sqrt{3}$ is irrational. Does a similar argument work to show $\sqrt{6}$ is irrational?}

Proof:To prove that $\sqrt{3}$ is irrational, we can use a proof by contradiction. The idea is to assume that $\sqrt{3}$ is rational, and then derive a contradiction.

Assumption: $\sqrt{3}$ is rational.

If $\sqrt{3}$ is rational, it can be expressed as a fraction $\frac{a}{b}$ in its simplest form, where $a$ and $b$ are integers with no common factors other than 1 (i.e., they are coprime), and $b$ is not equal to 0.

So, we have: $\sqrt{3}=\frac{a}{b}$

Squaring both sides of the equation: $3=\frac{a^{2}}{b^{2}}$

Rearranging the equation: $a^{2}=3 \cdot b^{2}$

From this equation, we can see that $a^{2}$ is a multiple of 3 , which implies that $a$ must also be a multiple of 3. Let's represent $a$ as $3 k$ where $k$ is an integer:

$$
\begin{gathered}
(3 k)^{2}=3 \cdot b^{2} \\
9 k^{2}=3 \cdot b^{2} \\
3 k^{2}=b^{2}
\end{gathered}
$$

Now, $b^{2}$ is also a multiple of 3 , which means $b$ must be a multiple of 3 . However, this contradicts our initial assumption that $a$ and $b$ have no common factors other than 1 (i.e., they are coprime). This contradiction arises from assuming that $\sqrt{3}$ is rational.

Hence, our initial assumption that $\sqrt{3}$ is rational must be false, which means that $\sqrt{3}$ is irrational. Now, to address whether a similar argument works for $\sqrt{6}$ :

No, a similar argument does not work for $\sqrt{6}$. The proof for the irrationality of $\sqrt{3}$ relies on the fact that it leads to a contradiction with respect to the common factors of $a$ and $b$. However, for $\sqrt{6}$, this argument does not apply directly because the factors of 6 are not the same as the factors of 3. In fact, $\sqrt{6}$ is not irrational; it is a rational number. It can be expressed as the fraction $\frac{2 \sqrt{3}}{3}$, which is a ratio of two integers.


\end{document}

