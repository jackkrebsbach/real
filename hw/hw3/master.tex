\documentclass{report}

\input{preamble}
\input{macros}
\input{letterfonts}

\title{Real Analysis HW \#3}
\author{Jack Krebsbach }

\date{Sep 20th}

\begin{document}

\maketitle

\qs{}{
1. Let $S$ be a nonempty set of real numbers that is bounded above and let $\beta=\sup S$. Suppose that $\beta \notin S$. Prove that for each $\epsilon>0$, the set $\{x \in S: x>\beta-\epsilon\}$ is infinite.
}

\qs{}{
2. Prove that the union of a countable set and an uncountable set is uncountable.
}

\qs{}{
3. Exercise 1.5.4:
}

(a) Show $|(a, b)|=|\mathbb{R}|$.

\par \bigskip
(b) Show that an unbounded interval like $(a, \infty)=\{x: x>a\}$ has the same cardinality at $\mathbb{R}$ as well.

\par \bigskip
(c) Show that $[0,1)$ has the same cardinality as $(0,1)$.

\qs{}{
4. Exercise 1.5.6:

}
\sol


 (a) Give an example of a countable collection of disjoint open intervals.
\par \bigskip

\par \bigskip

 (b) Give an example of an uncountable collection of disjoint open intervals, or argue thatno such collection exists.


\qs{}{
5. ChatGPT Challenge: Find an example of a sequence of closed bounded intervals $I_1, I_2, \ldots$ with the property that $\bigcap_{k=1}^n I_k \neq \emptyset$ for all $n \in \mathbb{N}$, but $\bigcap_{n=1}^{\infty} I_n=\emptyset$
}

Proof: Let $I_n=\left[\frac{1}{n}, 1-\frac{1}{n}\right]$ for $n \in \mathbb{N}$.
For each natural number $n$, the interval $I_n$ is a closed bounded interval, and their intersection is non-empty:
$$
\bigcap_{k=1}^n I_k=\left[\frac{1}{n}, 1-\frac{1}{n}\right] \neq \emptyset
$$
However, when we consider the infinite intersection:
$$
\bigcap_{n=1}^{\infty} I_n=\emptyset
$$
The infinite intersection is empty because as $n$ approaches infinity, the left endpoint $\frac{1}{n}$ approaches zero, and the right endpoint $1-\frac{1}{n}$ approaches 1 . So, the infinite intersection becomes the empty set.
\end{document}

