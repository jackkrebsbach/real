\documentclass{report}

\input{preamble}
\input{macros}
\input{letterfonts}
\newcommand{\contra}{
$\rightarrow\!\leftarrow$
}

\title{Real Analysis HW \#3}
\author{Jack Krebsbach }

\date{Sep 20th}

\begin{document}

\maketitle

\qs{}{
1. Let $S$ be a non-empty set of real numbers that is bounded above and let $\beta=\sup S$. Suppose that $\beta \notin S$. Prove that for each $\epsilon>0$, the set $\{x \in S: x>\beta-\epsilon\}$ is infinite.
}

\sol
\begin{myproof}
    
Assume, by way of contradiction, that $A = \{x \in S: x>\beta-\epsilon\}$ is finite. If $A$ is finite then let $\alpha = \max A$, with $\alpha \in A.$ Since $\alpha$ is an upper bound for $A$ and $\beta$ is supremun of $A$ we know that $\beta \leq \alpha$ and $\alpha \leq \beta.$ Thus, $\beta = \alpha \in A$ \contra. We know $\beta \not\in A.$ Thus, $A$ must be infinite.

\end{myproof}
\qs{}{
2. Prove that the union of a countable set and an uncountable set is uncountable.
}
\begin{myproof}
  Assume, by way of contradiction, that the union of a countable set $A$ and a uncountable set $B$ is countable. Then, $A \cup B$ is countable. By Theorem 1.5.7, then $B \subset A \cup B$ is countable or finite \contra. This is a problem because we know $B$ is uncountable. Thus, $A \cup B$ must be uncountable. 

  
\end{myproof}
\qs{}{
3. Exercise 1.5.4:
}

(a) Show $|(a, b)|=|\mathbb{R}|$. 


\sol
$$
f(x) = \tan\left(\frac{\pi}{b-a}\left[x - \frac{a+b}{2}\right]\right)
$$
We have domain $f$ is $(a,b)$, the range is $(-\infty, \infty) = \RR$ . The function $f$ is both onto and 1-1. Thus, $(a,b) \sim \RR.$

\par \bigskip
(b) Show that an unbounded interval like $(a, \infty)=\{x: x>a\}$ has the same cardinality at $\mathbb{R}$ as well.

\sol
$$
f(x)=log(x-a)
$$
This is a 1-1 onto function of the real numbers. The domain of $f$ is $(a, \infty)$ and the co-domain is $\RR.$

\par \bigskip
(c) Show that $[0,1)$ has the same cardinality as $(0,1)$. 


\bigskip
\sol
Let the domain of $f$ be $[0,1).$

$$ f(x) = 
  \begin{cases} 
    \frac{1}{2} & x= 0 \\
    \frac{x}{2} &  x \in \{ \frac{1}{2^n} \colon n \in \NN \} \\
    x & x \neq 0, x \not\in \{ \frac{1}{2^n} \colon n \in \NN \}
   \end{cases}
$$
\qs{}{
4. Exercise 1.5.6:
}
\sol


 (a) Give an example of a countable collection of disjoint open intervals.
\par \bigskip

$$A=\{(n, n+1): n \in \NN\} = \{(1,2),(2,3),(3,4),\dots\}$$
$$
f \colon \NN \rightarrow A 
$$ $$ f(n)= (n, n+1)$$
\par \bigskip

 (b) Give an example of an uncountable collection of disjoint open intervals, or argue that no such collection exists.


No such collection exists.

\begin{myproof}
    
Let $X$ be a collection of disjoint intervals. Each interval $x =(a,b)$ where $a,b \in \RR$ and $x\in X$ is open, and thus must contain at least two real numbers. By the density of $\QQ$ in $\RR$ we know that there exists $r \in \QQ$ such that $a < r < b.$ Thus, we obtain a new set of rational numbers, $Y \subset \QQ$.

Note that while each $a, b$ may not be unique between intervals, each $r$ is, as it lies between the bounds of the \textit{disjoint} open intervals. Combining Theorem 1.5.6 ($\QQ$ is countable) and Theorem 1.5.7 (A subset of a countable set is either countable or finite) $Y$ must be countable. Hence, any collection of disjoint open intervals is countable.

\end{myproof}

\qs{}{
5. ChatGPT Challenge: Find an example of a sequence of closed bounded intervals $I_1, I_2, \ldots$ with the property that $\bigcap_{k=1}^n I_k \neq \emptyset$ for all $n \in \mathbb{N}$, but $\bigcap_{n=1}^{\infty} I_n=\emptyset$
}

Proof: Let $I_n=\left[\frac{1}{n}, 1-\frac{1}{n}\right]$ for $n \in \mathbb{N}$.
For each natural number $n$, the interval $I_n$ is a closed bounded interval, and their intersection is non-empty:
$$
\bigcap_{k=1}^n I_k=\left[\frac{1}{n}, 1-\frac{1}{n}\right] \neq \emptyset
$$
\textbf{
  Your proposition seems reasonable at first but be careful with your intervals.  When $n=1$, we have $I_n = \left[1,0 \right]$ which is the empty set. Thus, any intersection after that must be the empty set and your example does not hold.}
  \par
  \textbf{You may consider $\mathbf{I_n=\left[\frac{1}{n+1}, 1-\frac{1}{n+1} \right]}$, but you still have problems. The next interval is $\mathbf{\left[\frac{1}{2}, \frac{1}{2}\right]}$, and every interval after that is a superset of this - the infinite intersection can not be the empty set.
}
\par\bigskip
However, when we consider the infinite intersection:
$$
\bigcap_{n=1}^{\infty} I_n=\emptyset
$$
The infinite intersection is empty because as $n$ approaches infinity, the left endpoint $\frac{1}{n}$ approaches zero, and the right endpoint $1-\frac{1}{n}$ approaches 1 . 

\par
\textbf{
Even if we consider my suggested intervals, this is still false. We have that  $$\mathbf{ \frac{1}{2} \in  \bigcap_{n=1}^{\infty} I_n}.$$ Make sure to do some sanity checks while you are going through examples. Make it concrete. Your logic does not hold. 
}

So, the infinite intersection becomes the empty set.
\end{document}

