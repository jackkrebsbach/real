\documentclass{report}

\input{preamble}
\input{macros}
\input{letterfonts}
\newcommand{\contra}{
$\rightarrow\!\leftarrow$
}

\title{Real Analysis HW \#4}
\author{Jack Krebsbach }

\date{Sep 20th}

\begin{document}

\maketitle
\qs{}{
Let $A$ be a nonempty bounded set. The maximum value is a number $x \in A$ such that $a \leq x$ $\forall a \in A$. Prove that a nonempty bounded set has a maximum value if and only if it contains its supremum.
}

\sol 

\begin{myproof}
    
\smallskip  
$\Rightarrow$ If a nonempty bounded set has a maximum value $x= Max(A)$ then for all $a \in A$ we have $ a \leq x.$ Then $x$ is an upperbound of the set $A.$ Let $b= \sup A.$ Then $b \leq x$  and $b \geq a$ for all $a \in A$. Thus, $a \leq b \leq x.$ Therefore the suprememum of $A$, $b$, is contained within the set; that is $b \in A.$  

$\Leftarrow$
Let $b = \sup A$ with $b \in A.$ Then for all $a \in A$ we have that $a \leq b.$ Thus, $b$ is the maximum of $A.$ 

\end{myproof}

\qs{}{
Let $A$ be a non-empty set and let $\mathcal{P}(\mathcal{A})$ represent the collection of all subsets of $A$; this set is known as the power set of $\mathrm{A}$.
}

(a) Suppose that $A$ has $n$ elements. Prove that $\mathcal{P}(\mathcal{A})$ has $2^n$ elements.
\bigskip
\par
\sol

Let $n=1.$ Take the set $A$ with $1$ element to be denoted set $A_1=\{ a_1 \}$ and the power set $P_1 = \mathcal{P}(\mathcal{A}_1)= \{\{a_1\}, \emptyset \}.$ Then $|P_1| = |\mathcal{P}(\mathcal{A}_1)|= 2^n  = 2^1 = 2.$  We have shown that this works for $n$. If $
|A| = n$ then $|\mathcal{P}(\mathcal{A})| = 2^{n}.$ We would like to show, through proof by induction, that this works for $n+1.$ \par
Consider $$\mathcal{P}(\mathcal{A}_{n+1}) = \mathcal{P}(\mathcal{A}_{n}) \cup  \{ \{p_{ni}\cup {a_{n+1}}\} \colon i  \in \NN_{2^n}\}$$  where $p_{ni}$ denotes the $ith$ element of the power set of $A_n.$ Because the right side of the union is disjoint from the left side of the union we can add the cardinalities together.

Hence, $$ |\mathcal{P}(\mathcal{A}_{n+1})| = |\mathcal{P}(\mathcal{A}_{n})| + |\{ \{p_{ni}\cup {a_{n+1}}\} \colon i  \in \NN_{2^n}\}|.$$
Then,
$$ |\mathcal{P}(\mathcal{A}_{n+1})| = 2^n +2^n= 2^n2^1 = 2^{n+1}.$$ Thus, $|\mathcal{P}(\mathcal{A}_{n+1})| = 2^{n+1}.$ We have shown that $\mathcal{P}(\mathcal{A})$ has $2^n$ elements.

\bigskip
(b) Suppose that $A$ is countable. Prove that $\mathcal{P}(\mathcal{A})$ is uncountable.
\bigskip
\par
\sol

Assume, by way of contradiction, that $\mathcal{P}(\mathcal{A})$ is countable. Then there exists an onto function $f\colon \NN \rightarrow \mathcal{P}(\mathcal{A}).$ Construct $B = \{n : n \not\in f(n)\}.$  Since $f$ is onto there must exist some $n_0 \in \NN$ such that $f(n_0) = B.$

We consider two cases:
\begin{enumerate}
  \item $n_0 \in B.$ Then by construction of the set $B$, $n_0 \not\in f(n_o).$ However, $f(n_0)=B$ so $n_0 \not\in B $ \contra.
  \item $n_0 \not\in B$. So $n_0 \in f(n_0).$ This implies $n_0 \in B$ \contra
\end{enumerate}

Thus, $f$ can not be onto. If it is not onto then $|\NN| \not= |\mathcal{P}(\mathcal{A})|$ and thus $\mathcal{P}(\mathcal{A})$ can not be countable.
\bigskip

(c) Suppose that $A$ is uncountable. Prove that there is no bijection between $A$ and $\mathcal{P}(\mathcal{A})$.
\bigskip
\begin{myproof}
    
Assume, by way of contradiction, that there exists a bijection between $A$ and $\mathcal{P}(\mathcal{A}),$ hence $f\colon A \rightarrow \mathcal{P}(\mathcal{A}).$ Construct $B = \{a : a \not\in f(n)\}.$  Since $f$ is onto there must exist some $a_0 \in \NN$ such that $f(a_0) = B.$

We consider two cases:
\begin{enumerate}
  \item $a_0 \in B.$ Then by construction of the set $B$, $a_0 \not\in f(a_o).$ However, $f(a_0)=B$ so $a_0 \not\in B $ \contra.
  \item $a_0 \not\in B$. So $a_0 \in f(a_0).$ This implies $a_0 \in B$ \contra
\end{enumerate}

Thus, there can not be a bijection  between  $A$ and $\mathcal{P}(\mathcal{A}).$ 

\end{myproof}

\qs{}{
Let $E$ be the set of all $x \in[0,1]$ whose decimal expansion contains only the digits 4 and 7 . Is $E$ countable? Explain.
}
\sol 
\begin{myproof}
  $E$ is not countable. Assume, by way of contradiction, that $E$ is countable.  Then there exists a bijection from the natural numbers to $E.$ Then we can list the correspondence where each $a_{nm} \in \{0,4,7\}.$

 \begin{centering}
     
\begin{tabular}{c c c c c c c c c c c}
 $\mathbf{N}$ & & $E$ & & & & & & & & \\
\hline 1 & $\longleftrightarrow$ & $f(1)$ & $=$ & $.a_{11}$ & $a_{12}$ & $a_{13}$ & $a_{14}$ & $a_{15}$ & $a_{16}$ & $\cdots$ \\
 2 & $\longleftrightarrow$ & $f(2)$ & $=$ &.$a_{21}$ & $a_{22}$ & $a_{23}$ & $a_{24}$ & $a_{25}$ & $a_{26}$ & .. \\
 3 & $\longleftrightarrow$ & $f(3)$ & $=$ &.$a_{31}$ & $a_{32}$ & $a_{33}$ & $a_{34}$ & $a_{35}$ & $a_{36}$ & .. \\
 4 & $\longleftrightarrow$ & $f(4)$ & $=$ &.$a_{41}$ & $a_{42}$ & $a_{43}$ & $a_{44}$ & $a_{45}$ & $a_{46}$ & $\ldots$ \\
 5 & $\longleftrightarrow$ & $f(5)$ & $=$ &.$a_{51}$ & $a_{52}$ & $a_{53}$ & $a_{54}$ & $a_{55}$ & $a_{56}$ & \dots \\
 6 & $\longleftrightarrow$ & $f(6)$ & $=$ & $.a_{61}$ & $a_{62}$ & $a_{63}$ & $a_{64}$ & $a_{65}$ & $a_{66}$ & $\cdots$ \\
$\vdots$ & & & & $\vdots$ & $\vdots$ & $\vdots$ & $\vdots$ & $\vdots$ & : & \\

\end{tabular}
\end{centering}   

Construct $E_0 = .e_1e_2e_3...$ where 
$$e_n =  \begin{cases} 
  4 &  a_{nn} = 7, 0\\
  7 & a_{nn} = 4
   \end{cases}
   $$

Thus, $E_0$ can not in the range of the surjection. Therefore $E$ is uncountable.
    
\end{myproof}

\qs{}{
Consider the function $h$ defined by
$$
h(x)=\left\{\begin{array}{cc}
0 & \text { if } x \notin \mathbb{Q} \\
1 / q & \text { if } x=p / q
\end{array}\right.
$$
Here it is assumed that the rational number $p / q$ is in reduced form and that $q>0$.
}
\bigskip
(a) Find $h(n)$ for each integer $n$

\sol For each integer $n$, $f(n)=1$
\bigskip

(b) Find three solutions to the equation $h(x)=1 / 3$.

\sol $ x = \frac{1}{3}, \frac{2}{3}, \frac{4}{3}$

\bigskip
(c) Find all of the solutions to the equation $h(x)=1 / 7$ that lie in the interval $(3,4)$.

\sol $x = \frac{22}{7}$, $\frac{23}{7}$, $\frac{24}{7}$, $\frac{25}{7}$, $\frac{26}{7}$, $\frac{27}{7}$,

\bigskip
(d) Prove that the set of all solutions to the equation $h(x)=1 / 5$ is countable infinite.

\sol
\begin{myproof}
    Let $S$ be the set of solutions to the equation $h(x)= \frac{1}{5}.$ Then for every $s \in S$ we can write as the ratio of two integers - a rational number. Then $S \subset \QQ.$ Since $S$ is a subset of a countable set (the rational numbers) then it must be countable (Theorem 1.5.7.) 
\end{myproof}
\bigskip
(e) Let $(a, b)$ be any interval and let $\epsilon>0$. Prove that $\{x \in(a, b): h(x) \geq \epsilon\}$ is a finite set.
\par
\bigskip
\sol
If $f(x) \geq \epsilon > 0$ then $x$ must be a rational number by construction of $h(x).$  Consider the solution set for every $N\in \NN$ where $x = \frac{p}{q}$ and $\frac{1}{N} = \frac{1}{q}$ under the constraint that $x \in (a,b)$.  Each of these sets is finite. For each $N$, $ a < \frac{p}{N} < b \implies aN < p < bN.$ So the set of numerators of $x$ at each $N$ is the integers between $aN$ and $bN.$ This is finite if we only consider reduced forms of $x.$

By Archy :) we can choose the minimum $N$ such that $\frac{1}{N+1} < \epsilon$ and $\frac{1}{N} \geq \epsilon.$ To find the number of elements in $A$ count the number of \textit{reduced form solutions} satisfying $h(x) = \frac{1}{n}$ for $n = 1,2,\dots, N$. Thus, $A$ is finite.

\pagebreak
\qs{}{
 ChatGPT Challenge: Find a sequence is bounded below, not bounded above, and contains an infinite number of negative terms.

}

Proof: Certainly, here's an example of a sequence that is bounded below, not bounded above, and contains an infinite number of negative terms:
Consider the sequence defined as follows: $-1,-2,-3,-4,-5, \ldots$
This sequence consists of negative integers and is strictly decreasing.

\par
\textbf{This is correct.}

It is bounded below by negative infinity (there is no lower bound as it goes to negative infinity), 
\par
\textbf{
  This doesn't mean anything. Infinity is not a number so this does not make sense. In addition, the problem asks for as sequence that \textit{is} bounded from below and you have just stated that there is no lower bound. Consider the function $f(n) =e^{n(-1)^n} -1$ where $n \in N$. The co domain of this function is bounded below by $-1$ but is not bounded from above. Does this satisfy the other conditions of the sequence?
}

not bounded above (it becomes more negative without bound), and it contains an infinite number of negative terms.

\textbf{Make sure that you are precise in your definitions.
Remember a set $A \subseteq \mathbf{R}$ is bounded above if there exists a number $b \in \mathbf{R}$ such that $a \leqslant b$ for all $a \in A$.  Similarly, the set $A$ is bounded below if there exists a lower bound $l \in \mathbf{R}$ satisfying $l \leqslant a$ for every $a \in A$.}

\end{document}

