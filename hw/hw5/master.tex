\documentclass{report}

\input{preamble}
\input{macros}
\input{letterfonts}
\newcommand{\contra}{
$\rightarrow\!\leftarrow$
}

\title{Real Analysis HW \#5}
\author{Jack Krebsbach }

\date{Oct 15th}

\begin{document}

\maketitle
\qs{}{

1. Let $x_n \geq 0$ for all $n \in \mathbb{N}$.
}

\bigskip
(a) If $\left(x_n\right) \rightarrow 0$, show that $\sqrt{x_n} \rightarrow 0$.


\bigskip
(b) If $\left(x_n\right) \rightarrow x$, show that $\sqrt{x_n} \rightarrow \sqrt{x}$.

\qs{}{

2. Let $\left(a_n\right)$ and $\left(b_n\right)$ be sequences of real numbers.
}

(a) Show that if $\left(a_n\right) \rightarrow 0$ and $\left(a_n-b_n\right) \rightarrow 0$, then $\left(b_n\right) \rightarrow 0$.

\bigskip
(b) Show that if $\left(a_n\right) \rightarrow 0$ and $\left|b_n-b\right| \leq a_n$, then $\left(b_n\right) \rightarrow b$


\qs{}{

3. Consider $y_1=1, y_{n+1}=\left(2 y_n+3\right) / 4$ for all $n \in \mathbb{N}$. Show by direct calculations that $y_1<y_2<2$. Then, show that if $y_{n-1}<y_n<2$ that $y_n<y_{n+1}<2$. Use this to show that $\left\{y_n\right\}$ converges and find its limit.

}

\qs{}{

4. (Cesaro Means). Show that if $\left(x_n\right)$ is a convergent sequence, then the sequence given by the averages:
$$
y_n=\frac{x_1+x_2+\ldots x_n}{n}
$$
also converges to the same limit.

}
\qs{}{

5. ChatGPT Challenge: To prove that if $\lim \left(a_n-b_n\right)=0$, then $\lim a_n=\lim b_n$, you can use the definition of a limit and the properties of limits. Here's a proof:

}


Proof:
Let $\epsilon>0$ be arbitrary. Since $\lim \left(a_n-b_n\right)=0$, there exists an $N_1$ such that for all $n \geq N_1$, we have $\left|a_n-b_n\right|<\epsilon / 2$. Now, consider the inequality $\left|a_n-b_n\right|<\epsilon / 2$. We can add $\left|b_n\right|$ to both sides of this inequality: $\left|a_n-b_n\right|+\left|b_n\right|<\epsilon / 2+\left|b_n\right|$

By the triangle inequality, we know that $\left|a_n-b_n+b_n\right| \leq\left|a_n-b_n\right|+\left|b_n\right|$, so: $\left|a_n\right|<\epsilon / 2+\left|b_n\right|$ Now, we have the following inequalities: $\left|a_n-b_n\right|<\epsilon / 2$ (from our choice of $N_1$ )
$\left|a_n\right|<\epsilon / 2+\left|b_n\right|$
Now, choose $N_2$ such that for all $n \geq N_2$, we have $\left|b_n\right|<\epsilon / 2$. Let $N=\max \left(N_1, N_2\right)$. For $n \geq N$, we have: $\left|a_n-b_n\right|<\epsilon / 2$ (from the choice of $N_1$ )
$\left|b_n\right|<\epsilon / 2$ (from the choice of $N_2$ )
Now, let's use these inequalities to bound $\left|a_n\right|$ :
$\left|a_n\right|=\left|a_n-b_n+b_n\right| \leq\left|a_n-b_n\right|+\left|b_n\right|<\epsilon / 2+\epsilon / 2=\epsilon$
This shows that for all $n \geq N$, we have $\left|a_n\right|<\epsilon$, which means that $\lim a_n=0$.
Since $\epsilon$ was arbitrary, we have shown that for any $\epsilon>0$, there exists an $N$ such that for all $n \geq N,\left|a_n\right|<\epsilon$, which is the definition of $\lim a_n=0$.
Therefore, we have proved that if $\lim \left(a_n-b_n\right)=0$, then $\lim a_n=\lim b_n$.


\end{document}

