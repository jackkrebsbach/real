\documentclass{report}

\input{preamble}
\input{macros}
\input{letterfonts}
\newcommand{\contra}{
$\rightarrow\!\leftarrow$
}

\title{Real Analysis HW \#5}
\author{Jack Krebsbach }

\date{Oct 15th}

\begin{document}

\maketitle
\qs{}{

1. Let $x_n \geq 0$ for all $n \in \mathbb{N}$.
}

\bigskip
(a) If $\left(x_n\right) \rightarrow 0$, show that $\sqrt{x_n} \rightarrow 0$.

\begin{myproof}
\bigskip
For all $\epsilon >0$ there exists $N \in \NN$ such that for all $n> \NN$, $|x_n - 0| < \epsilon.$
This implies that $|x_n| < \epsilon.$ Let $\epsilon_0 = \epsilon^2.$

Then $$|x_n| < \epsilon_0 = \epsilon^2 \implies |\sqrt{x_n}| < \epsilon  \implies  |\sqrt{x_n} - 0| < \epsilon.$$ Thus, we have have shown $\sqrt{x_n} \rightarrow 0.$
\end{myproof}

\bigskip
(b) If $\left(x_n\right) \rightarrow x$, show that $\sqrt{x_n} \rightarrow \sqrt{x}$.


\bigskip
\begin{myproof}
We consider two cases: 
\begin{enumerate}
  \item If $x = 0$ then see proof of (a).
  \item If $x > 0.$  Since $(x_n) \rightarrow x$ we know that for every $\epsilon >0 $ there exists $N \in \NN$ such that for all $n > N$, $|x_n -x | < \epsilon.$ Let $\epsilon_0 = \epsilon \sqrt{x}.$

    Then $$\left|x_n -x  \right| < \epsilon \sqrt{x} \implies \left| \sqrt{x_n}^2 - \sqrt{x}^2  \right| < \epsilon \sqrt{x}.$$

This a difference of squares so, $$\left| \sqrt{x_n}^2 - \sqrt{x}^2  \right| = \left| (\sqrt{x_n} - \sqrt{x})(\sqrt{x_n} + \sqrt{x})  \right| < \epsilon \sqrt{x}$$
$$ \implies\left| (\sqrt{x_n} - \sqrt{x}) \left(\frac{\sqrt{x_n} + \sqrt{x}}{\sqrt{x}}\right)\right| < \epsilon$$
$$ \implies\left| (\sqrt{x_n} - \sqrt{x}) \left(\frac{\sqrt{x_n}}{\sqrt{x}} + 1 \right)\right| < \epsilon$$

Since $\left(\frac{\sqrt{x_n}}{\sqrt{x}} + 1\right) > 1$ it must be that  $\left|\sqrt{x_n} - \sqrt{x}\right| < \epsilon.$ Thus, $\sqrt{x_n} \rightarrow \sqrt{x}.$
\end{enumerate}
\end{myproof}

\qs{}{
2. Let $\left(a_n\right)$ and $\left(b_n\right)$ be sequences of real numbers.
}
(a) Show that if $\left(a_n\right) \rightarrow 0$ and $\left(a_n-b_n\right) \rightarrow 0$, then $\left(b_n\right) \rightarrow 0$.
\begin{myproof}
    
  If $\left(a_n\right) \rightarrow 0$ then for all $\epsilon >0$ there exists $N_1$ such that for all $n> N_1$, $|a_n - 0| < \frac{\epsilon}{2}.$ Similarly, for all $\epsilon >0$ there exists $N_2$ such that for all $n > N_2$, $| (a_n - b_n) - 0 |< \frac{\epsilon}{2}.$ Consider the sum of these two quantities with $n >N^\star = \max\{N_1,N_2\}.$ Then 
$$| a_n  - 0 |+ | (a_n - b_n) - 0 |<  \frac{\epsilon}{2} + \frac{\epsilon}{2}$$ and it follows

$$| 0 - a_n |+ |a_n - b_n|<  \epsilon.$$

By the Triangle Inequality Theorem, 

$$|0 - a_n  + a_n - b_n| = |0 - b_n| \leq | 0 - a_n |+ |a_n - b_n|< \epsilon.$$

Thus, $|0 - b_n| = |b_n - 0 |< \epsilon \implies (b_n) \rightarrow 0.$

\end{myproof}


\bigskip
(b) Show that if $\left(a_n\right) \rightarrow 0$ and $\left|b_n-b\right| \leq a_n$, then $\left(b_n\right) \rightarrow b$

\begin{myproof}
    
  If $(a_n) \rightarrow 0 $ then for all $\epsilon > 0 $ there exists
$N \in \NN$ such that for all $n > N$, $|a_n - 0| < \frac{\epsilon}{2}.$ We know that $$|b_n - b| \leq |a_n|  \implies |b_n - b| - |a_n|\leq 0 < \frac{\epsilon}{2}.$$

After the summing the two quantities,
 $$|b_n - b| - |a_n| + |a_n -0| < \frac{\epsilon}{2} + \frac{\epsilon}{2}$$ $$\implies |b_n - b| - |a_n| + |a_n| < \epsilon$$
$$ \implies |b_n - b|  < \epsilon $$ $$\implies (b_n) \rightarrow b.$$

\end{myproof}

\qs{}{

3. Consider $y_1=1, y_{n+1}=\left(2 y_n+3\right) / 4$ for all $n \in \mathbb{N}$. Show by direct calculations that $y_1<y_2<2$. Then, show that if $y_{n-1}<y_n<2$ that $y_n<y_{n+1}<2$. Use this to show that $\left\{y_n\right\}$ converges and find its limit.

}

\begin{myproof}

  If $y_1 = 1$ then $y_{1+1} =  [2(1) +3]/4 = \frac{5}{4} < 2.$ So $y_1 < y_2 < 2.$
    
  If $y_{n-1}<y_n<2$ then $$y_{n-1}<y_{n}<2 \implies 2y_{n-1}<2y_{n}<2(2)\implies 2y_{n-1} + 3<2y_{n} +3 <2(2) +3  $$
 $$ \implies \frac{2y_{n-1} + 3}{4}<\frac{2y_{n} +3}{4} <\frac{2(2) +3}{4}. $$
 $$ \implies y_{n} <y_{n+1} <\frac{7}{4} < 2$$

We see that $y_n < y_{n+1}$ so the sequence is monotone and bounded by $2.$  Therefore, it is convergent.
\par
Let $\lim_{n \rightarrow \infty} y_{n} = L$. Consider $$\lim_{n \rightarrow \infty} y_{n+1} = \lim_{n \rightarrow \infty}\left(2 y_n+3\right) / 4$$ 
$$\implies L = \left(2 L+3\right) / 4$$ 

$$\implies 4L = 2 L+3$$ 

$$\implies 2L = 3$$ 

$$\implies L = 3/2.$$ 

Thus, ${y_n} \rightarrow 3/2.$
\end{myproof}
\pagebreak

\qs{}{

4. (Cesaro Means). Show that if $\left(x_n\right)$ is a convergent sequence, then the sequence given by the averages:
$$
y_n=\frac{x_1+x_2+\ldots x_n}{n}
$$
also converges to the same limit.

}
\bigskip
\begin{myproof}
    
  Let $(x_n)$ converge to $x$. If $(x_n)$ is convergent then for every $\epsilon >0$ there exists $N \in \NN$ such that for all $n > N$, $| x_{n} - x| < \epsilon.$

  Then $$ \left| \frac{x_1 + x_1 +\cdots + x_{n} }{n}  - x\right|= $$$$ \left| \frac{x_1 + x_1 + \cdots + x_{N}+ \cdots + x_{n} }{n}  - x\right|= $$
$$ \left| \frac{x_1 + x_1 + \cdots + x_{N}}{n} + \frac{ x_{N+1}+ \cdots + x_{n} }{n} - x\right| \leq \underbrace{\left| \frac{x_1 + x_1 + \cdots + x_{N}}{n} \right|}_{K}+ \underbrace{ \left|\frac{ x_{N + 1}+ \cdots + x_{n} }{n} - x\right|}_{Z}$$

The last step is a result of the Triangle Inequality Theorem. We know that for all the terms where $n>N$, $| x_{n} - x| < \epsilon.$ Then the average of of all $x_n$ where $n>N$, $ \bar{x}_n =\frac{ x_{N + 1}+ \cdots + x_{n} }{n-N},$ will also be have the property $|\bar{x}_n - x| < \epsilon$.  As $n$ tends to infinity the expression $\frac{ x_{N + 1} + \cdots + x_{n} }{n}$  will tend towards $ \bar{x}_n =\frac{ x_{N + 1}+ \cdots + x_{n} }{n-N}.$ Therefore, we can choose a large enough $n$ such that $Z < \epsilon.$ We also know as $n \rightarrow \infty $, $K \rightarrow 0.$

Hence, when $n$ is sufficiently large, $K + Z <\epsilon$
$  \implies\left| \frac{x_1 + x_1 +\cdots + x_{n} }{n}  - x\right|< \epsilon$ and therefore $(y_n) \rightarrow x$ in addition to $(x_n) \rightarrow x.$
\bigskip
\end{myproof}

\qs{}{

5. ChatGPT Challenge: To prove that if $\lim \left(a_n-b_n\right)=0$, then $\lim a_n=\lim b_n$, you can use the definition of a limit and the properties of limits. Here's a proof:

}


Proof:
Let $\epsilon>0$ be arbitrary. Since $\lim \left(a_n-b_n\right)=0$, there exists an $N_1$ such that for all $n \geq N_1$, we have $\left|a_n-b_n\right|<\epsilon / 2$. Now, consider the inequality $\left|a_n-b_n\right|<\epsilon / 2$. We can add $\left|b_n\right|$ to both sides of this inequality: $\left|a_n-b_n\right|+\left|b_n\right|<\epsilon / 2+\left|b_n\right|$

By the triangle inequality, we know that $\left|a_n-b_n+b_n\right| \leq\left|a_n-b_n\right|+\left|b_n\right|$, so: $\left|a_n\right|<\epsilon / 2+\left|b_n\right|$ Now, we have the following inequalities: $\left|a_n-b_n\right|<\epsilon / 2$ (from our choice of $N_1$ )
$\left|a_n\right|<\epsilon / 2+\left|b_n\right|$
\par
\textbf{Everything up until this point tracks. You have clearly shown how we got to this point.}
\par
Now, choose $N_2$ such that for all $n \geq N_2$, we have $\left|b_n\right|<\epsilon / 2$. 
\par
\textbf{Where does this come from? This only works if we know $(b_n)$ converges to 0. You are trying to prove that $(b_n)$ and $(a_n)$ converge to the same value. We do not know if one or both converge to zero.}

Let $N=\max \left(N_1, N_2\right)$. For $n \geq N$, we have: $\left|a_n-b_n\right|<\epsilon / 2$ (from the choice of $N_1$ )
$\left|b_n\right|<\epsilon / 2$ (from the choice of $N_2$ )
Now, let's use these inequalities to bound $\left|a_n\right|$ :
$\left|a_n\right|=\left|a_n-b_n+b_n\right| \leq\left|a_n-b_n\right|+\left|b_n\right|<\epsilon / 2+\epsilon / 2=\epsilon$
This shows that for all $n \geq N$, we have $\left|a_n\right|<\epsilon$, which means that $\lim a_n=0$.
Since $\epsilon$ was arbitrary, we have shown that for any $\epsilon>0$, there exists an $N$ such that for all $n \geq N,\left|a_n\right|<\epsilon$, which is the definition of $\lim a_n=0$.
Therefore, we have proved that if $\lim \left(a_n-b_n\right)=0$, then $\lim a_n=\lim b_n$.
\par
\textbf{Again we are trying to show that $(a_n)$ and $(b_n)$ converge to the same number. You have not even proved that $\lim b_n = 0$, so this does not show that $\lim (a_n) = \lim (b_n)$. Make sure you have a clear goal of what you are trying to prove and lay clear logical steps to prove that assertion.}


\end{document}

