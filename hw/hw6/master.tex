\documentclass{report}

\input{preamble}
\input{macros}
\input{letterfonts}
\newcommand{\contra}{
$\rightarrow\!\leftarrow$
}

\title{Real Analysis HW \#6}
\author{Jack Krebsbach }

\date{Nov 1st}

\begin{document}

\maketitle

\qs{}{
Ex 2.4 .2
}

(a) Consider the recursively defined sequence $y_1=1, Y_{n+1}=3-y_n$ and set $y=\lim y_n$. Because $\left(y_n\right)$ and $\left(y_{n+1}\right)$ have the same limit, taking the limit across the recursive equation gives $y=3-y$. Solving for $y$, we conclude $\lim y_n=3 / 2$. What is wrong with this argument?

\bigskip
\sol

This argument is does not work because the sequence is not convergent. This strategy to find the limit of the recursive sequence only works when we know the sequence is convergent. We have that $x_1 = 1$, $x_2 = 2$, $x_3 = 1$, and then the sequence repeats, never converging to a value.

\bigskip

\bigskip
(b) This time set $y_1=1$ and $y_{n+1}=3-\frac{1}{y_n}$. Can the strategy in (a) be applied to compute the limit of this sequence? Why or why not. Justify your answer.

\bigskip
\sol 

Yes, we can apply this strategy. 

Consider the fact that $y_2 = 2$. Then $y_2 > y_1$. We want to show then that if $y_{n+1} > y_{n}$ we have $y_{n+2} > y_{n+1}$. 

Then
$$y_{n+1}>y_n $$ $$ \implies -y_{n+1}<-y_{n} $$ $$ \implies-\frac{1}{y_{n+1}}>-\frac{1}{y_n} $$ $$ \implies3-\frac{1}{y_{n+1}}>3-\frac{1}{y_n} $$ $$ \implies y_{n+2}>y_{n+1}$$

Thus, the sequence is increasing. The sequence is also bounded from above by 3. By the Monotone Convergence Theorem $(x_n)$ is convergent. Therefore, we can apply the strategy to compute the limit.


\bigskip
\bigskip

\qs{}{

For each natural number $n$, let
$$
x_n=\frac{1}{n+1}+\frac{1}{n+2}+\cdots+\frac{1}{2 n} .
$$

Prove that the sequence $\left(x_n\right)$ converges.
}

\begin{myproof}

We first see that the sequence is increasing and monotone because 

$$x_{n+1} - x_{n}= 
\left(\frac{1}{n+2}+\cdots+\frac{1}{2 n} + \frac{1}{2n+1} + \frac{1}{2n+2} \right) - \left(\frac{1}{n+1}+\frac{1}{n+2}+\cdots+\frac{1}{2 n}\right)$$  

$$ =  \frac{1}{2n+1} + \frac{1}{2n+2} - \frac{1}{n+1}$$

$$ =  \frac{1}{2n+1} + \frac{1}{2n+2} - \frac{2}{2n+2} > 0.$$



Because the first term is the largest of $n$ terms in the partial sum   the following inequality holds:
$$ \sum_{k =n}^{2n} \frac{1}{k+1} < n\left(\frac{1}{n+1}\right).$$

Then we have
$$ \sum_{k =n}^{2n} \frac{1}{k+1} < n\left(\frac{1}{n+1}\right) = \left(\frac{n}{n+1}\right) =\left(1 - \frac{1}{n + 1}\right) \leq 1.$$

Thus, $(x_n)$ is increasing monotone and bounded above by 1. By the Monotone Convergence Theorem $(x_n)$ is convergent.

\end{myproof}



\qs{}{

  Consider the sequence
$$
x_n=1+\frac{1}{\sqrt{2}}+\frac{1}{\sqrt{3}}+\cdots+\frac{1}{\sqrt{n}}-2 \sqrt{n} .
$$

Show that $\left(x_n\right)$ converges.
}

\begin{myproof}
  First we show that $(x_n)$ is decreasing and monotone:
$$
\begin{aligned}
  x_{n+1} & -x_n=\left(1+\cdots+\frac{1}{\sqrt{n+1}}-2 \sqrt{n+1}\right)-\left(1+\cdots+\frac{1}{\sqrt{n}} - 2 \sqrt{n}\right) \\
& =\frac{1}{\sqrt{n+1}}-2 \sqrt{n+1}+2 \sqrt{n} \\
& =\frac{1}{\sqrt{n+1}}-2[\sqrt{n+1}-\sqrt{n}]\left[\frac{\sqrt{n+1}+\sqrt{n}}{\sqrt{n+1}+\sqrt{n}}\right] \\
& =\frac{2}{2\sqrt{n+1}}-\frac{2}{\sqrt{n+1}+\sqrt{n}}<0
\end{aligned}
$$

So, $\left(x_n\right)$ is decreasing and bounded from above by $x_1 = 1.$
We now show that $(x_n)$ is bounded from below. Consider,

$$\frac{2}{2 \sqrt{n}} > \frac{2}{\sqrt{n+1} + \sqrt{n}}$$

and can be simplified by multiplying by the conjugate of the denominator of the right hand side,
 $$\frac{2}{2 \sqrt{n}} > 2\sqrt{n+1} - 2\sqrt{n}.$$
 

 So,
 $$
\left(\sum_{k=1}^{n} \frac{1}{\sqrt{k}} \right)-2 \sqrt{n} > \left( \sum_{k=1}^{n} 2\sqrt{k+1} - 2\sqrt{k} \right) - 2\sqrt{n}
 $$

The right hand side of the inequality is telescoping:
 $$
 2\sqrt{2} \hspace{0.5em}\boxed{-2} + 2\sqrt{3}- 2\sqrt{2} + 2\sqrt{4} - 2\sqrt{3}+ \cdots + 2\sqrt{n+1} - 2\sqrt{n}
$$

 $$
 = -2  + 2\sqrt{n+1}- 2\sqrt{n}
$$

 $$
 = -2  + 2(\sqrt{n+1}- \sqrt{n}) > -2
$$
Therefore,

$$\left(\sum_{k=1}^{n} \frac{1}{\sqrt{k}} \right)-2 \sqrt{n} > -2.$$

Thus, showing that $(x_n)$ is bounded from below by $-2$. By the Monotone Convergence Theorem $(x_n)$ must be convergent.

\end{myproof}

\qs{}{

Prove that $\left(x_n\right)$ converges where $x_n=\sum_{k=1}^n k^2 2^{-k}$.
}

\begin{myproof}

  First we see that when $k=16$,  $$\frac{1}{k^2} = \frac{1}{16^2} = \frac{16^2}{16^4} = \frac{16^2}{(2^4)^4} = \frac{16^2}{2^{16}} = \frac{k^2}{2^k}$$

Consider the following expressions in anticipation of showing that for $k>16$, $$k^22^k > k^4 \implies \frac{1}{k^22^k}  < \frac{1}{k^4} \implies \frac{k^2}{2^k}  < \frac{1}{k^2}.$$

Let $k=(16+n):$
$$
(16+n)^2 2^{(16+n)} \text { and }(16+n)^4.
$$

Taking the natural log of both expressions yields
$$
\ln \left((16+n)^22^{(6+n)}\right) \text { and } \ln (16+n)^4.
$$

Then,
$$
\begin{aligned}
& \implies\ln \left((16+n)^2\right)+\ln \left(2^{(16+n)}\right) \text { and } 4 \ln (16+n) \\
& \implies2 \ln (16+n)+(16+n) \ln (2) \text { and } 4 \ln (16+n)
\end{aligned}
$$

Subtracting $2 \ln(16+n)$ from both expressions
$$
\begin{aligned}
&\implies (16+n) \ln (2) \text { and } 2\ln (16+n) \\
&\implies 16 \ln (2)+n \ln (2) \text { and }  2\ln (16+n).
\end{aligned}
$$

When $n=1$ we have that
$$
11.78... > 11.332...
$$
Since the left hand side increases linearly by an amount of $\ln(2) \approx  0.69 $ and the right hand side increases at a maximum of  $\ln(17) - \ln(16) \approx 0.06$ we can conclude that for all $k>16$, we have $k^22^k > k^4 \implies \frac{k^2}{2^k} < \frac{1}{k^2}$ .

Then, we can make the following comparison:  
$$\sum_{k=17}^n \frac{k^2}{2^k} \leq \sum_{k=17}^n \frac{1}{k^2}\leq \sum_{k=17}^n \frac{1}{k(k-1)}  = \sum_{k=17}^n \frac{1}{(k-1)} - \frac{1}{k}= \frac{1}{16} - \frac{1}{17}  + \frac{1}{17} - \frac{1}{18} + \cdots - \frac{1}{k-1} + \frac{1}{k-1} - \frac{1}{n} \leq \frac{1}{16} - \frac{1}{n} \leq \frac{1}{16}.$$


Let $B= \sum_{k=1}^{16} \frac{k^2}{2^k}.$ Then, $\lim_{n \rightarrow \infty}(x_n) \leq |B| + \frac{1}{16}.$ Thus, $(x_n)$ is bounded from above by $|B| + \frac{1}{16}.$

Consider 
$$x_{n+1} - x_{n}=\sum_{k=1}^{n+1} k^2 2^{-k} - \sum_{k=1}^n k^2 2^{-k} = (n+1)^2 2^{-(n+1)} >0.$$ Thus, $(x_n)$ is increasing and monotone. By the Monotone Convergence Theorem $(x_n)$ is convergent.
\bigskip

\end{myproof}

\bigskip
\pagebreak
\qs{}{

Consider the sequence $x_n=\sum_{k=1}^n \ln \left(\frac{k+1}{k}\right)$. Does the sequence converge? Explain.
}


\begin{myproof}

  No, the sequence does not converge to a finite number. To show that $(x_n)$ converges to infinity first write the sum as a telescoping series:

$$ x_n = \ln(2) + \ln(3/2) + \ln(4/3) + \cdots + \ln\left( \frac{n}{n-1}\right) + \ln\left(\frac{n +1}{n}\right) $$

$$  = \ln(2) + \ln(3) - \ln(2) + \ln(4) - \ln(3) + \cdots + \ln(n) - \ln(n-1) + \ln(n+1) - \ln(n) $$

$$  =  \ln(n+1).$$

Then


$$ \lim_{n \rightarrow   \infty} (x_n)= \lim_{n \rightarrow \infty} \ln(n+1) = \infty.$$

Thus, $(x_n)$ converges to positive infinity.


\end{myproof}

\bigskip
\qs{}{
CHATBOT Challenge: Consider the sequence
$$
x_n=1+\frac{1}{\sqrt{2}}+\frac{1}{\sqrt{3}}+\cdots+\frac{1}{\sqrt{n}}-2 \sqrt{n} .
$$

Show that $\left(x_n\right)$ converges.
}


To show that the sequence $\left\{x_n\right\}$ converges, we can use the limit properties of sequences. We will prove that the limit exists by finding the limit of $x_n$ as $n$ approaches infinity.

The sequence $x_n$ is defined as:
$$
x_n=1+\frac{1}{\sqrt{2}}+\frac{1}{\sqrt{3}}+\ldots+\frac{1}{\sqrt{n}}-2 \sqrt{n}
$$

We want to find $\lim _{n \rightarrow \infty} x_n$. First, let's break it down into two parts and consider their limits separately:



\begin{enumerate}

  \item The $\operatorname{sum} \frac{1}{\sqrt{2}}+\frac{1}{\sqrt{3}}+\ldots+\frac{1}{\sqrt{n}}$.
    \item The term $-2 \sqrt{n}$.
    
\end{enumerate}

For the first part, we can use the fact that the limit of a sum is the sum of the limits (if they exist):
$$
\lim _{n \rightarrow \infty}\left(\frac{1}{\sqrt{2}}+\frac{1}{\sqrt{3}}+\ldots+\frac{1}{\sqrt{n}}\right)=\lim _{n \rightarrow|\infty|} \frac{1}{\sqrt{2}}+\lim _{n \rightarrow \infty} \frac{1}{\sqrt{3}}+\ldots+\lim _{n \rightarrow \infty} \frac{1}{\sqrt{n}}
$$

Each term in the sum approaches zero as $n$ goes to infinity because $\frac{1}{\sqrt{n}}$ converges to zero:

\textbf{This is incorrect. Look at each term, the only term that depends on n is the last term $\frac{1}{\sqrt{n}}.$ This last term certainly goes to zero, but the terms before it do not. }
$$
\lim _{n \rightarrow \infty} \frac{1}{\sqrt{n}}=0
$$

So, the limit of the sum of these terms is:
$$
\lim _{n \rightarrow \infty}\left(\frac{1}{\sqrt{2}}+\frac{1}{\sqrt{3}}+\ldots+\frac{1}{\sqrt{n}}\right)=0
$$
\textbf{Look at my comment above, this is not true. This limit actually converges to infinity, not 0.}

For the second part, the limit of $-2 \sqrt{n}$ as $n$ approaches infinity is also straightforward:
$$
\lim _{n \rightarrow \infty}(-2 \sqrt{n})=-\infty
$$

Now, we can find the limit of the entire sequence $x_n$ by combining the limits of its two parts:
$$
\lim _{n \rightarrow \infty} x_n=\lim _{n \rightarrow \infty}\left(1+\frac{1}{\sqrt{2}}+\frac{1}{\sqrt{3}}+\ldots+\frac{1}{\sqrt{n}}-2 \sqrt{n}\right)=0-\infty=-\infty
$$

\textbf{This incorrect. Even if we use the fact that part (1) converges to infinity this would an indeterminate form. We need another strategy to solve this problem}

Since $\lim _{n \rightarrow \infty} x_n$ is a finite value $(-\infty)$, we can conclude that the sequence $\left\{x_n\right\}$ converges to $-\infty$.

\textbf{This is not true, infinity is not a finite value! It is true  $(x_n)$ converges to a finite value, but you have not shown this. Hint: first see if the sequence is monotone. Next, try and find a telescoping partial sum that you can show bounds $(x_n).$ }
\end{document}

