\documentclass{report}

\input{preamble}
\input{macros}
\input{letterfonts}
\newcommand{\contra}{
$\rightarrow\!\leftarrow$
}

\title{Real Analysis H \#6}
\author{Jack Krebsbach }

\date{Nov 1st}

\begin{document}

\maketitle

\qs{}{
Ex 2.4 .2
}

(a) Consider the recursively defined sequence $y_1=1, Y_{n+1}=3-y_n$ and set $y=\lim y_n$. Because $\left(y_n\right)$ and $\left(y_{n+1}\right)$ have the same limit, taking the limit across the recursive equation gives $y=3-y$. Solving for $y$, we conclude $\lim y_n=3 / 2$. What is wrong with this argument?

\bigskip
\sol 

The problem is that we have not shown that it is bounded from above and from below. We only know that the sequence converges if it is monotone in addition to being bounded. 


\bigskip
(b) This time set $y_1=1$ and $y_{n+1}=3-\frac{1}{y_n}$. Can the strategy in (a) be applied to compute the limit of this sequence? Why or why not. Justify your answer.

\bigskip
\sol 

Yes, we can apply this because we see that the sequence is bounded above by 3 and bounded below as well.


\bigskip

\qs{}{

For each natural number $n$, let
$$
x_n=\frac{1}{n+1}+\frac{1}{n+2}+\cdots+\frac{1}{2 n} .
$$

Prove that the sequence $\left(x_n\right)$ converges.
}

\begin{myproof}

Consider the integration from $n$ to $2n$:
 
$$ \lim_{n \rightarrow \infty} \int_{n}^{2n} \left( \frac{1}{1+x} dx  \right)$$
$$= \lim_{n \rightarrow \infty} \left(\ln x+1 ]_{n}^{2n} \right)$$

$$= \lim_{n \rightarrow \infty} \left(\ln (2n+1) - \ln (n+1) \right)$$
$$= \lim_{n \rightarrow \infty} \ln \left(\frac{2n+1}{n+1}\right)$$

$$=  \ln 2.$$

We can graphically show that summing $(x_n)$ with the rectangle method has a smaller area than the integration above. Thus, since the integral converges to a number, $(x_n)$ must as well. 

\end{myproof}


\pagebreak

\qs{}{

  Consider the sequence
$$
x_n=1+\frac{1}{\sqrt{2}}+\frac{1}{\sqrt{3}}+\cdots+\frac{1}{\sqrt{n}}-2 \sqrt{n} .
$$

Show that $\left(x_n\right)$ converges.
}

\begin{myproof}
    To show that $(x_n)$ converges we will show that it is bounded below and above. 

    Consider $$\frac{2}{2 \sqrt{n}} < \frac{2}{\sqrt{n+1} + \sqrt{n}}.$$

Multiplying the right hand side by the denominators conjugate yields
 $$\frac{2}{2 \sqrt{n}} < \frac{2}{\sqrt{n+1} + \sqrt{n}} \left(\frac{\sqrt{n+1} - \sqrt{n}}{\sqrt{n+1} - \sqrt{n}}  \right) = 2\sqrt{n+1} + 2\sqrt{n}.$$

 Therefore, we can show that $(x_n)$ is bounded above by the following:
 $$\frac{2}{2 \sqrt{n}} < 2\sqrt{n+1} + 2\sqrt{n} $$ implies  that 

 $$
\left(\sum_{k=1}^{n} \frac{1}{\sqrt{k}} \right)-2 \sqrt{n} < \left( \sum_{k=1}^{k} 2\sqrt{k+1} + 2\sqrt{k} \right) - 2\sqrt{n}
 .$$

 The left hand side is a telescoping series:

 $$
\left(\sum_{k=1}^{n} \frac{1}{\sqrt{k}} \right)-2 \sqrt{n} < 2 + 2\sqrt{2} - 2 + 2\sqrt{3}- 2\sqrt{2} + \cdots + 2\sqrt{n-1} + 2\sqrt{n} - 2 \sqrt{n-1} - 2\sqrt{n}
$$
and we find that
 $$
\left(\sum_{k=1}^{n} \frac{1}{\sqrt{k}} \right)-2 \sqrt{n} < 0 $$

We similarly show that $(x_n)$ is bounded from below.

$$\frac{2}{2 \sqrt{n}} > \frac{2}{\sqrt{n-1} + \sqrt{n}}$$

and can be simplified to
 $$\frac{2}{2 \sqrt{n}} > 2\sqrt{n-1} - 2\sqrt{n}.$$
 

 So,


 $$
\left(\sum_{k=1}^{n} \frac{1}{\sqrt{k}} \right)-2 \sqrt{n} > \left( \sum_{k=1}^{n} 2\sqrt{k+1} - 2\sqrt{k} \right) - 2\sqrt{n}
 $$

Again, the right hand side of the inequality is telescoping:
 $$
 2\sqrt{2} \hspace{0.5em}\boxed{-2} + 2\sqrt{3}- 2\sqrt{2} + \cdots + 2\sqrt{n-1} + 2\sqrt{n} - 2 \sqrt{n-1} - 2\sqrt{n}
$$

 and we are left with 

$$\left(\sum_{k=1}^{n} \frac{1}{\sqrt{k}} \right)-2 \sqrt{n} > -2.$$

Thus showing that $(x_n)$ is bounded from above and from below. By the Monotone Convergence theorem $(x_n)$ must be convergent.

\end{myproof}

\qs{}{

Prove that $\left(x_n\right)$ converges where $x_n=\sum_{k=1}^n k^2 2^{-k}$.
}
\qs{}{

Consider the sequence $x_n=\sum_{k=1}^n \ln \left(\frac{n+1}{n}\right)$. Does the sequence converge? Explain.
}
\qs{}{
ChatGPT Challenge: Consider the sequence
$$
x_n=1+\frac{1}{\sqrt{2}}+\frac{1}{\sqrt{3}}+\cdots+\frac{1}{\sqrt{n}}-2 \sqrt{n} .
$$

Show that $\left(x_n\right)$ converges.
}

To show that the sequence $\left\{x_n\right\}$ converges, we can use the limit properties of sequences. We will prove that the limit exists by finding the limit of $x_n$ as $n$ approaches infinity.

The sequence $x_n$ is defined as:
$$
x_n=1+\frac{1}{\sqrt{2}}+\frac{1}{\sqrt{3}}+\ldots+\frac{1}{\sqrt{n}}-2 \sqrt{n}
$$

We want to find $\lim _{n \rightarrow \infty} x_n$. First, let's break it down into two parts and consider their limits separately:
1. The $\operatorname{sum} \frac{1}{\sqrt{2}}+\frac{1}{\sqrt{3}}+\ldots+\frac{1}{\sqrt{n}}$.
2. The term $-2 \sqrt{n}$.

For the first part, we can use the fact that the limit of a sum is the sum of the limits (if they exist):
$$
\lim _{n \rightarrow \infty}\left(\frac{1}{\sqrt{2}}+\frac{1}{\sqrt{3}}+\ldots+\frac{1}{\sqrt{n}}\right)=\lim _{n \rightarrow|\infty|} \frac{1}{\sqrt{2}}+\lim _{n \rightarrow \infty} \frac{1}{\sqrt{3}}+\ldots+\lim _{n \rightarrow \infty} \frac{1}{\sqrt{n}}
$$

Each term in the sum approaches zero as $n$ goes to infinity because $\frac{1}{\sqrt{n}}$ converges to zero:
$$
\lim _{n \rightarrow \infty} \frac{1}{\sqrt{n}}=0
$$

So, the limit of the sum of these terms is:
$$
\lim _{n \rightarrow \infty}\left(\frac{1}{\sqrt{2}}+\frac{1}{\sqrt{3}}+\ldots+\frac{1}{\sqrt{n}}\right)=0
$$

For the second part, the limit of $-2 \sqrt{n}$ as $n$ approaches infinity is also straightforward:
$$
\lim _{n \rightarrow \infty}(-2 \sqrt{n})=-\infty
$$

Now, we can find the limit of the entire sequence $x_n$ by combining the limits of its two parts:
$$
\lim _{n \rightarrow \infty} x_n=\lim _{n \rightarrow \infty}\left(1+\frac{1}{\sqrt{2}}+\frac{1}{\sqrt{3}}+\ldots+\frac{1}{\sqrt{n}}-2 \sqrt{n}\right)=0-\infty=-\infty
$$

Since $\lim _{n \rightarrow \infty} x_n$ is a finite value $(-\infty)$, we can conclude that the sequence $\left\{x_n\right\}$ converges to $-\infty$.


\end{document}

