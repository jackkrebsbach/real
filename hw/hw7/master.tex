\documentclass{report}

\input{preamble}
\input{macros}
\input{letterfonts}
\newcommand{\contra}{
$\rightarrow\!\leftarrow$
}

\title{Real Analysis CW \#7}
\author{Jack Krebsbach }

\date{Nov 8th}

\begin{document}

\maketitle
\qs{}{

Let $\left(x_n\right)$ be a sequence and suppose that the sequence $\left(x_{n+1}-x_n\right)$ converges to 0 . Give an example to show that the sequence $\left(x_n\right)$ may not converge. (See CHATBOT Challenge)
}

\sol
Let $$x_n= \sum_{k=1}^{n} 1/k.$$

This is the harmonic series, which converges to infinity. Let $\epsilon>0.$ By Archimedes Principle there exists $N \in \NN$ such that $1/N < \epsilon.$ Let $n>N.$ Then $$|(x_{n+1} - x_n) -0 | = \left|\frac{1}{n+1} - \frac{1}{n}\right| < \left|\frac{1}{N} \right| < \epsilon.$$

Thus, $\left(x_{n+1}-x_n\right)$ converges to 0, but $(x_n)$ converges to infinity.

\bigskip
\qs{}{

Let $\left(x_k\right)$ and $\left(y_k\right)$ be two sequences and let $\left(r_k\right)$ be a sequence of positive numbers that converges to 0 . Suppose that $0<\left|y_k-x_k\right|<r_k \forall k \in \mathbb{N}$.
}

\bigskip
(a) Give an example to show that the sequences $\left(x_k\right)$ and $\left(y_k\right)$ may not converge.
\bigskip

\sol
Let $$y_k = k + \frac{1}{k}$$ and let $$x_k = k +\frac{1}{(k+1)}.$$

\bigskip
\bigskip
\bigskip
(b) Suppose that $\left(x_k\right)$ converges to $L$. Prove that the sequence $\left(y_k\right)$ converges to $L$.

\begin{myproof}
    Let $\epsilon >0.$ Because $(r_k)$ converges to $0$ then there exists $N_1 \in \NN$ such that for all $k>N_1$, $\left|r_k - 0 \right| < \frac{\epsilon}{2}.$

    Because $(x_k)$ converges to $L$ there exists $N_2 \in \NN$ such that for all $k>N_2,$ $\left| x_k -L\right| < \frac{\epsilon}{2}.$ Let $N = \max\{N_1,N_2\}$ and $k> N.$

    Then, $$0 < \left| y_k - x_k \right| < \left|r_k -0\right| < \frac{\epsilon}{2}.$$
$$\implies \left| y_k - L + L - x_k \right| < \frac{\epsilon}{2}.$$

$$\implies \left| y_k - L\right|  - \left|L - x_k \right| \leq \left| y_k - L + L - x_k \right|< \frac{\epsilon}{2}.$$

$$\implies \left| y_k - L\right|  - \underbrace{\left|L - x_k \right|}_{< \frac{\epsilon}{2}} < \frac{\epsilon}{2}.$$

And we know that $\left| x_k -L\right| < \frac{\epsilon}{2}$,

$$\implies \left| y_k - L\right| < \epsilon$$

Thus, $y_k$ converges to $L.$
\end{myproof}


\bigskip
\qs{}{

Assume that $\left(x_n\right)$ is a bounded sequence with the property that every convergent subsequence of $\left(x_n\right)$ converges to the same limit $x \in \mathbb{R}$. Show that $\left(x_n\right)$ must converge to $x$.
}

\bigskip
\bigskip
\qs{}{

Let $\left(x_n\right)$ be a Cauchy sequence. Show directly that $\left(x_n\right)$ is bounded.
}

Let $\epsilon >0$ and $(x_n)$ be a Cauchy sequence. Then there exists $N \in \NN$ such that for all $n,m \geq N$, $\left| x_n - x_m\right| < 1.$ 

Then, 
 $$ \left|x_m - x_N\right| < 1.$$

$$ \implies \left| x_m \right| - \left|x_N\right| < \left| x_m - x_N\right| < 1.$$


$$ \implies \left| x_m \right| - \left|x_N\right|  < 1.$$

$$ \implies \left| x_m \right| < 1 + \left|x_N\right|.$$


Then, we have a bound for all the terms $N$ and beyond. Let $$ B = \max\{|x_1|, |x_2|, |x_3|, \dots, |x_N| + 1\}.$$

Then $|x_n| < B$ for all $n \in \NN$ and therefore $(x_n)$ must be bounded.

\bigskip

\qs{}{

Let $\left(x_n\right)$ be a sequence and let $r$ be a real number satisfying $0<r<1$. Suppose that $\left|x_{n+1}-x_n\right|<$ $r\left|x_n-x_{n-1}\right|$ for all $n>2$. Prove that $\left(x_n\right)$ is a Cauchy sequence and hence convergent.
}

Let $\epsilon >0.$ Since $\left|x_{n+1}-x_n\right|<$ $r\left|x_n-x_{n-1}\right|$ there exists $N \in \NN$ such that for all $m>n>N$ we have $r|x_n - x_n-1| < \frac{r^N}{(m-n)}\epsilon.$

Then, $$\left|x_{n+1}-x_n\right|< r\left|x_n-x_{n-1}\right| < \frac{r^N}{(m-n)}\epsilon$$

$$\implies \left|x_{n+1}-x_n\right|< \frac{r^N}{(m-n)}\epsilon$$

for all $m > n > N.$ 

Then,

$$|x_m - x_n|= |x_m - x_{m-1} + x_{m-1} - x_{m-2}  + x_{m-2} - x_{m-3}+ \dots+ x_{n+1} -  x_{n}| $$

$$\leq |x_m - x_{m-1}| + |x_{m-1} - x_{m-2}|  + 
|x_{m-2} -  x_{m-3}| + \dots + | x_{n+1} -  x_{n}| $$

$$< r^N(m-n)| x_{n+1} -  x_{n}| <r^{2N} \epsilon < \epsilon.$$

Hence, $(x_n)$ is Cauchy, and thus convergent.

\pagebreak
\qs{}{

CHAT GPT CHALLENGE:
Let $\left(x_n\right)$ be a sequence such that $\left(x_{n+1}-x_n\right)$ converges to 0 . We want to show that $\left(x_n\right)$ also converges.
}
By the definition of a convergent sequence, we need to show that for any $\epsilon>0$, there exists a positive integer $N$ such that for all $n \geq N,\left|x_n-L\right|<\epsilon$ for some real number $L$.

\textbf{Good job setting the stage for your proof and clearly stating what you need to show to substantiate your claim. This is correct.}

Since $\left(x_{n+1}-x_n\right)$ converges to 0 , we can choose $\epsilon>0$ and find a positive integer $M$ such that for all $m \geq M,\left|\left(x_{m+1}-x_m\right)-0\right|<\epsilon$. This implies that for all $m \geq M,\left|x_{m+1}-x_m\right|<\epsilon$.
Now, let $N=M$. For any $n \geq N$, we have:
$$
\begin{aligned}
\left|x_n-x_N\right| & =\left|\left(x_n-x_{n-1}\right)+\left(x_{n-1}-x_{n-2}\right)+\ldots+\left(x_{N+1}-x_N\right)\right| \\
& \leq\left|x_n-x_{n-1}\right|+\left|x_{n-1}-x_{n-2}\right|+\ldots+\left|x_{N+1}-x_N\right| \\
& <\epsilon+\epsilon+\ldots+\epsilon \quad\left(\text { Since }\left|x_{m+1}-x_m\right|<\epsilon \text { for all } m \geq M\right) \\
& =(n-N) \epsilon .
\end{aligned}
$$

Dividing both sides by $(n-N)$, we get:
$$
\left|x_n-x_N\right|<\epsilon
$$
\textbf{This is incorrect. When you divide both sides by $(n-N)$
you get
$$
\frac{1}{(n-N)}\left|x_n-x_N\right|<\epsilon
$$
which does not show that $\left|x_n-x_N\right|<\epsilon. $
}

This shows that for any $\epsilon>0$ and $n \geq N$, we have $\left|x_n-x_N\right|<\epsilon$, where $N=M$. Therefore, $\left(x_n\right)$ converges to $x_N$.

\textbf{This does not prove what you wanted to show. }

Hence, we have shown that if $\left(x_{n+1}-x_n\right)$ converges to 0 , then $\left(x_n\right)$ also converges.

\end{document}
