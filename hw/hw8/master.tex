\documentclass{report}

\input{preamble}
\input{macros}
\input{letterfonts}
\newcommand{\contra}{
$\rightarrow\!\leftarrow$
}

\title{Real Analysis HW \#8}
\author{Jack Krebsbach }

\date{Nov 15th}

\begin{document}
\maketitle


\dfn{Fibonacci Sequence}{
The Fibonacci sequence, $1,2,3,5,8,13, \ldots$ is given by the recursive formula
$$
F_{n+1}=F_n+F_{n-1}
$$
where $F_1=1$ and $F_2=2$. Let $a_n=\frac{F_n}{F_{n-1}}$.
}

\qs{}{

Suppose that $\left\{a_n\right\}$ converges to a limit. What must that limit be? Hint: Divide the above equation by $F_n$ to find an equation relating $a_{n+1}$ to $a_n$.

}

\qs{}{

  Show that $\frac{3}{2} \leq a_n \leq 2 \text{  } \forall n \geq 2$.
}
\qs{}{

For each $n>2$, prove that $\left|a_{n+1}-a_n\right| \leq\left(\frac{2}{3}\right)^2\left|a_n-a_{n-1}\right|$.
}
\qs{}{

Prove that for each $m \geq 2,\left|a_{m+1}-a_m\right| \leq\left(\frac{2}{3}\right)^{2(m-2)}\left|a_3-a_2\right|$.
}

\qs{}{

Use the inequality in (4) to show that $\left\{a_n\right\}$ is a Cauchy sequence and therefore converges to a limit.
}
\qs{}{
CHAT GPT CHALLENGE:
To prove that a bounded sequence $\left(x_n\right)$ with the property that every convergent subsequence of $\left(x_n\right)$ converges to the same limit $x \in \mathbb{R}$ must converge to $x$, you can use the following proof:
}

Let $\epsilon>0$ be given. Since $\left(x_n\right)$ is bounded, it has a limit point, say $y$, which means that there exists a subsequence $\left(x_{n_k}\right)$ that converges to $y$. Since every convergent subsequence of $\left(x_n\right)$ converges to $x$, we have $y=x$.

Now, consider $\epsilon>0$ and $y=x$. Since $\left(x_n\right)$ is a bounded sequence, it has a limit point, and there exists a subsequence $\left(x_{n_k}\right)$ that converges to $x$.

By the definition of the limit, there exists an $N_1$ such that for all $k \geq N_1,\left|x_{n_{\star}}-x\right|<\epsilon / 2$.
Additionally, because every convergent subsequence of $\left(x_n\right)$ converges to $x$, we can choose another $N_2$ such that for all $k \geq N_2,\left|x_{n_k}-x\right|<\epsilon / 2$.

Now, let $N=\max \left(N_1, N_2\right)$. For all $n \geq N$, there exists $k_1 \geq N$ and $k_2 \geq N$ such that:
$$
\begin{aligned}
& \left|x_{n_{k_1}}-x\right|<\frac{\epsilon}{2} \\
& \left|x_{n_{k_2}}-x\right|<\frac{\epsilon}{2}
\end{aligned}
$$

By the triangle inequality:
$$
\begin{aligned}
\left|x_n-x\right| & \leq\left|x_n-x_{n_{k_1}}\right|+\left|x_{n_{k_1}}-x\right| \\
& <\frac{\epsilon}{2}+\frac{\epsilon}{2} \\
& =\epsilon
\end{aligned}
$$

So, for all $n \geq N,\left|x_n-x\right|<\epsilon$, which means that $\left(x_n\right)$ converges to $x$.
Therefore, we've shown that if $\left(x_n\right)$ is a bounded sequence with the property that every convergent subsequence converges to the same limit $x \in \mathbb{R}$, then $\left(x_n\right)$ itself must converge to $x$.
\end{document}
