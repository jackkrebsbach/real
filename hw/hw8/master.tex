\documentclass{report}

\input{preamble}
\input{macros}
\input{letterfonts}
\newcommand{\contra}{
$\rightarrow\!\leftarrow$
}

\title{Real Analysis HW \#8}
\author{Jack Krebsbach }

\date{Nov 15th}

\begin{document}
\maketitle


\dfn{Fibonacci Sequence}{
The Fibonacci sequence, $1,2,3,5,8,13, \ldots$ is given by the recursive formula
$$
F_{n+1}=F_n+F_{n-1}
$$
where $F_1=1$ and $F_2=2$. Let $a_n=\frac{F_n}{F_{n-1}}$.
}

\qs{}{

Suppose that $\left\{a_n\right\}$ converges to a limit. What must that limit be? Hint: Divide the above equation by $F_n$ to find an equation relating $a_{n+1}$ to $a_n$.

}
\sol
From the recursive formula, dividing by $F_n$ yields:

$$\frac{F_{n+1}}{F_n} = 1 + \frac{F_{n-1}}{F_n} $$

Then, $$a_{n+1} = 1 + \frac{F_{n-1}}{F_n}$$
$$ \implies a_{n+1} = 1 + \frac{1}{a_n}$$

Let $L = \lim_{n \rightarrow \infty} a_n,$ then  $$ L = 1 + \frac{1}{L}$$
$$ \implies L^2 = L + 1$$


$$\implies L^2 - L -1=0.$$

By the quadratic formula,  $$L = \frac{ 1 \pm \sqrt{5}}{2}.$$ Since this sequence is positive for all $n \in \NN$  we want the positive solution. Thus, $$L  = \frac{ 1 + \sqrt{5}}{2}.$$



\pagebreak

\qs{}{

  Show that $\frac{3}{2} \leq a_n \leq 2 \text{  } \forall n \geq 2$.
}

\begin{myproof}
    
  Let $n\in \NN.$  We have that $a_1 = 1$, $a_2 = 2$, $a_3 = 3/2.$ Thus, $$\frac{3}{2} \leq a_n \leq 2$$ for $1,2,4 \in \NN.$ We want to show that if this is true for $a_n$ this is also true for $a_{n+1}.$

We assume that $$ \frac{3}{2} \leq a_n \leq 2$$ is true.  Then,
$$ \frac{2}{3} \geq \frac{1}{a_n} \geq \frac{1}{2}$$

$$\implies  1+ \frac{2}{3} \geq 1+ \frac{1}{a_n} \geq 1+\frac{1}{2}$$

$$\implies  1+ \frac{1}{2} \leq 1+ \frac{1}{a_n} \leq 1+\frac{2}{3}$$
$$\implies  \frac{3}{2} \leq a_{n+1} \leq \frac{5}{3} < 2.$$

Thus, $\frac{3}{2} \leq a_n \leq 2$ for all  $n \geq 2.$ 
\end{myproof}


\qs{}{

For each $n>2$, prove that $\left|a_{n+1}-a_n\right| \leq\left(\frac{2}{3}\right)^2\left|a_n-a_{n-1}\right|$.
}
\begin{myproof}
    
Let $n>2$. Then $$ \left| a_{n+1}  - a_{n} \right| $$
$$= \left| 1 + \frac{1}{a_{n}}  - 1 - \frac{1}{a_{n-1}} \right| $$

$$= \left|\frac{1}{a_{n}}  - \frac{1}{a_{n-1}} \right| $$

$$= \left|\frac{ a_{n-1 } - a_n }{a_{n-1} a_n}   \right| $$

Since for all $n\geq2$ we have that $a_n \geq \frac{3}{2},$

$$ \left| a_{n+1}  - a_{n} \right| =\left|\frac{ a_{n-1 } - a_n }{a_{n-1} a_n}\right| \leq \left|\frac{ a_{n-1 } - a_n }{\left(\frac{3}{2}\right) \left(\frac{3}{2}\right)}\right|$$
$$ \implies \left| a_{n+1}  - a_{n} \right|  \leq \left(\frac{2}{3}\right)^2\left|a_{n-1 } - a_n \right|.$$

\end{myproof}

\pagebreak
\qs{}{

Prove that for each $m > 2,\left|a_{m+1}-a_m\right| \leq\left(\frac{2}{3}\right)^{2(m-2)}\left|a_3-a_2\right|$.
}

\sol
\begin{myproof}
    
We see $a_1 = 1$, $a_2 = 2$, $a_3 = 3/2,$ and $a_4 = 5/3$.  Thus, when $n=3$
$$\left|a_4 - a_3 \right| \leq \left(\frac{2}{3} \right)^2 \left|a_3 - a_2 \right|$$
$$\implies \left|\frac{5}{3} - \frac{3}{2} \right| \leq \left(\frac{2}{3} \right)^2 \left|\frac{3}{2} - \frac{4}{2} \right|$$

$$\implies \left|\frac{1}{6}  \right| \leq \left(\frac{2}{3} \right)^2 \left| - \frac{1}{2} \right|$$

$$\implies \left|\frac{1}{6}  \right| \leq \left|\frac{2}{9} \right|$$
$$\implies \left|\frac{9}{54}  \right| \leq \left|\frac{12}{54} \right|$$

Hence, the inequality holds. Since $\left|a_{n+1}-a_n\right| \leq\left(\frac{2}{3}\right)^2\left|a_n-a_{n-1}\right|$ for $n>2$ it follows that
$$\implies \left|a_{4} - a_3\right| \leq \left(\frac{2}{3}\right)^2 \left|a_3 -a_2\right|$$
$$\implies \left|a_{5} - a_4\right| \leq \left(\frac{2}{3}\right)^4 \left|a_3 -a_2\right|$$

$$\implies \left|a_{6} - a_5\right| \leq \left(\frac{2}{3}\right)^6 \left|a_3 -a_2\right|$$

and generally when $m>2,$
$$ \left|a_{m+1} - a_m\right| \leq \left(\frac{2}{3}\right)^{2(m-2)} \left|a_3 -a_2\right|.$$

\end{myproof}
\pagebreak
\qs{}{

Use the inequality in (4) to show that $\left\{a_n\right\}$ is a Cauchy sequence and therefore converges to a limit.
}

\begin{myproof}

Let $m>2$ and $$B =\left(\frac{3}{2}\right)^{4}\left|a_3 -a_2\right|.$$

We know that  
$$ \left|a_{m+1} - a_m\right| \leq \left(\frac{2}{3}\right)^{2(m-2)} \left|a_3 -a_2\right|.$$

$$ \implies \left|a_{m+1} - a_m\right| \leq \left(\frac{2}{3}\right)^{2m}B.$$
Let $\epsilon > 0.$ Choose $N\in\NN$ such that for all $n > N$, $$ B \frac{4}{5}\left(\frac{2}{3}\right)^{2n}< \epsilon.$$

Let $m > n > N.$ We have,

$$|a_m - a_n|= |a_m - a_{m-1} + a_{m-1} - a_{m-2}  + a_{m-2} - a_{m-3}+ \dots+ a_{n+1} -  a_{n}| $$

$$\leq |a_m - a_{m-1}| + |a_{m-1} - a_{m-2}|  + 
|a_{m-2} -  a_{m-3}| + \dots + | a_{n+1} -  a_{n}| $$

$$ \leq \sum_{k={n+1}}^{m} |a_k - a_{k-1}|$$
$$ \leq \sum_{k={n+1}}^{m} \left(\frac{2}{3}\right)^{2k}B  .$$


Now, if $(x_n)= \sum_{k={n+1}}^{m} \left(\frac{2}{3}\right)^{2k}B , $ then
$$x_n - \left(\frac{2}{3}\right)^2x_n = B\left[\sum_{k={n+1}}^{m}\left(\frac{2}{3}\right)^{2k} - \sum_{k={n+2}}^{m+1}\left(\frac{2}{3}\right)^{2k}\right]$$

$$\implies x_n\left(1 - \frac{4}{9}\right) = B\left[ \left(\frac{2}{3}\right)^{2(n+1)}- \left(\frac{2}{3}\right)^{2(m+1)}\right] $$

$$\implies x_n = B\left[\frac{\left(\frac{2}{3}\right)^{2(n+1)}- \left(\frac{2}{3}\right)^{2(m+1)}}{5/9}\right] $$

$$\leq B \frac{4}{5}\left(\frac{2}{3}\right)^{2n} < \epsilon.$$


Thus, $(x_n)$ is Cauchy and hence convergent.

    
\end{myproof}

\pagebreak
\qs{}{
CHAT GPT CHALLENGE:
To prove that a bounded sequence $\left(x_n\right)$ with the property that every convergent subsequence of $\left(x_n\right)$ converges to the same limit $x \in \mathbb{R}$ must converge to $x$, you can use the following proof:
}

Let $\epsilon>0$ be given. Since $\left(x_n\right)$ is bounded, it has a limit point, say $y$, which means that there exists a subsequence $\left(x_{n_k}\right)$ that converges to $y$. 

\textbf{
How do you know that $(x_n)$ converges at all? The Monotone Convergence Theorem only works if you know that the sequence is bounded and monotone. You can not assume that $(x_n)$ converges. You are correct in saying that there exists a subsequence that converges to a number. This is by the Bolzano Weierstrass
}
Since every convergent subsequence of $\left(x_n\right)$ converges to $x$, we have $y=x$.

Now, consider $\epsilon>0$ and $y=x$. Since $\left(x_n\right)$ is a bounded sequence, it has a limit point, and there exists a subsequence $\left(x_{n_k}\right)$ that converges to $x$.

By the definition of the limit, there exists an $N_1$ such that for all $k \geq N_1,\left|x_{n_{k}}-x\right|<\epsilon / 2$.
Additionally, because every convergent subsequence of $\left(x_n\right)$ converges to $x$, we can choose another $N_2$ such that for all $k \geq N_2,\left|x_{n_k}-x\right|<\epsilon / 2$.

Now, let $N=\max \left(N_1, N_2\right)$. For all $n \geq N$, there exists $k_1 \geq N$ and $k_2 \geq N$ such that:
$$
\begin{aligned}
& \left|x_{n_{k_1}}-x\right|<\frac{\epsilon}{2} \\
& \left|x_{n_{k_2}}-x\right|<\frac{\epsilon}{2}
\end{aligned}
$$

By the triangle inequality:
$$
\begin{aligned}
\left|x_n-x\right| & \leq\left|x_n-x_{n_{k_1}}\right|+\left|x_{n_{k_1}}-x\right| \\
& <\frac{\epsilon}{2}+\frac{\epsilon}{2} \\
& =\epsilon
\end{aligned}
$$

So, for all $n \geq N,\left|x_n-x\right|<\epsilon$, which means that $\left(x_n\right)$ converges to $x$.
Therefore, we've shown that if $\left(x_n\right)$ is a bounded sequence with the property that every convergent subsequence converges to the same limit $x \in \mathbb{R}$, then $\left(x_n\right)$ itself must converge to $x$.
\end{document}
