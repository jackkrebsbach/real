\documentclass{report}

\input{preamble}
\input{macros}
\input{letterfonts}
\newcommand{\contra}{
$\rightarrow\!\leftarrow$
}

\title{Real Analysis HW \#9}
\author{Jack Krebsbach }

\date{Nov 28th}

\begin{document}
\maketitle

\qs{}{

Let $g: A \rightarrow \mathbb{R}$ and $f: A \rightarrow \mathbb{R}$. Suppose that $\lim _{x \rightarrow a} f(x)=0$.
\bigskip
}

(a) Show that $\lim _{x \rightarrow a} f(x) g(x)=0$ for any function $g$ as above IS NOT TRUE.

\begin{myproof}
     Assume for the sake of contradiction that $\lim _{x \rightarrow a} f(x) g(x)=0$ for  any $g(x).$ Consider the case when $g(x)= 1/x^2.$ Let $\epsilon >0.$ Then there exists $\delta$ such that whenever $0 < |x - 0| < \delta$ we have $|f(x)g(x) -0|<\epsilon.$

  By the Archimedes principle there exists $N \in \NN$ such that for all $n > N,$ $0 < |1/n| < \delta.$l Thus, $$ |f(1/n)g(1/n)-0|=|f(1/n)n^2 -0|<\epsilon.$$

  Herein lies the absurdity, if $f$ approaches zero slower then $n^2$ increases, we  choose $n_* > N$ such that $ |f(1/n_*)n_*^2| > \epsilon \rightarrow\!\leftarrow$
Hence, $\lim _{x \rightarrow a} f(x) g(x)=0$ for any function $g$ is not true.
\end{myproof}

\bigskip
\bigskip
\bigskip

(b) Assume that $g$ is a bounded function on $A$. Show that $\lim _{x \rightarrow a} g(x) f(x)=0$.

\begin{myproof}
    Let $\epsilon >0$ and $g$ be bounded by $B \in \RR^+$. So $|g(x)| < B$ for all $x \in \RR.$ Because $\lim _{x \rightarrow a} f(x)=0$ then there exists $\delta$ such that if $c\in \RR$ and $0 < | x -c | <\delta$ we automatically have $|f(x) - 0| <  \epsilon / B$.
Now, $$ |g(x)f(x) - 0| < |g(x)|\left|\frac{\epsilon}{B}\right| \leq |B|\left|\frac{\epsilon}{B}\right| = \epsilon.  $$

\end{myproof}

\pagebreak
\qs{}{

Let $a$ and $b$ be real numbers with $a \neq 0$. Use the definition of continuity to prove that the function $f$ defined by $f(x)=a x+b$ is continuous at every real number.
}


\begin{myproof}
    Let $\epsilon >0$ and $c \in \RR.$ Choose $\delta = \epsilon/a.$ If we have $ |x-c|< \delta$ it follows $$ |f(x) - f(c)| = |ax +b - (ac+b)| = | ax + b -ac - b | = |ax -ac| = |a(x-c)| < |a \frac{\epsilon}{a} | < \epsilon.$$

Thus, $f(x)=a x+b$ is continuous at every real number.
\end{myproof}

\qs{}{
 Use the definition of limit to prove that $\lim _{x \rightarrow c} x^2=c^2$ for every real number $c$.
}

\begin{myproof}
  Let $\epsilon > 0.$ Choose $\delta = \min\{1, \epsilon/(2c+1)\}$. If we have $0 < |x-c| < \delta$ it follows that, $$|f(x) - L| = |x^2 - c^2| = |x + c||x-c| < (2c+1)\frac{\epsilon}{2c+1} = \epsilon$$.

Thus, $\lim _{x \rightarrow c} x^2=c^2$ for every real number $c$.
\end{myproof}

\qs{}{

Find constants $a$ and $b$ so that the function $f$ defined by
$$
f(x)=\left\{\begin{array}{cc}
3 a x^2+1 & 0 \leq x \leq 1 \\
a x+b & 1<x<2 \\
2 b x+a & 2 \leq x \leq 4
\end{array}\right.
$$
has a limit at each point of $[0,4]$. Be sure to show the limit exists.
}
\bigskip
\sol

First we find constants $a$ and $b$ so that $f(x)$ has a limit defined at each point $[0,4].$ Plugging in $1$ and $2$ in each of the equations defined in the piecewise function $f(x)$ yields a system of equations:$$ 2a +1 = a + b \implies 2a +1 = b$$
and $$2a +b = 4b + a \implies a = 3b.$$

Substituting $a = 3b$ into the first equation yields $$2(3b) = b \implies 6b + 1 =6 \implies 5b = -1 \implies b = \frac{-1}{5}.$$ Finally, solving for $a = 3b = 3 (-1/5) = -3/5.$ Thus, $$ b = \frac{-1}{5} \text { and } a = \frac{-3}{5} $$ and $f$ becomes


$$
f(x)=\left\{\begin{array}{cc}
-9/5 x^2+1 & 0 \leq x \leq 1 \\
-3/5 x - 1/5 & 1<x<2 \\
-2/5 x - 3/5 & 2 \leq x \leq 4
\end{array}\right.
$$
\qs{}{
Let $f:[a, b] \rightarrow \mathbb{R}$ be continuous on $[a, b]$ and suppose that $f(x)=0$ for each irrational number $x$ in $[a, b]$. Prove that $f(x)=0$ for all $x \in[a, b]$.

}
\begin{myproof}
    

If $f:[a, b] \rightarrow \mathbb{R}$ is continuous on $[a, b]$ then it must be continuous on every point $c \in [a,b].$ Assume, for the sake of contradiction, that there exists $x_0 \in [a,b]$ with $x_0 \in \RR \setminus \QQ$ and $f(x_0) \not =0.$

Then $f$ must be continuous at $x_0,$ and there must exist $\delta$ such that $|x-x_0| < \delta$ implies that $|f(x) - f(x_0)| < |f(x_0)|/2.$ 

By the density of rationals in the real numbers there must exist an irrational number, $x_I,$ such that $|x_I - x_0| < \delta.$ 

Since $x_I$ is irrational then $f(x_I) = 0$ and we have $|f(x_I) - f(x_0)| =  |f(x_0)| < \left|f(x_0)\right|/2 \implies 1 < 1/2 \rightarrow\!\leftarrow.$  Thus, we have arrived at a contradiction which means that $f(x)=0$ for all $x \in[a, b].$

\end{myproof}

\qs{}{

CHAT GPT CHALLENGE:
Let $f:[a, b] \rightarrow \mathbb{R}$ be a continuous function on $[a, b]$, and suppose that $f(x)=0$ for each rational number $x$ in $[a, b]$. We aim to show that $f(x)=0$ for all $x \in[a, b]$.
}

Assume, for the sake of contradiction, that there exists $c \in[a, b]$ such that $f(c) \neq 0$. Since $f$ is continuous, for any $\epsilon>0$, there exists a $\delta>0$ such that for all $x \in[a, b]$, if $|x-c|<\delta$, then $|f(x)-f(c)|<\epsilon$.

Consider the case when $\epsilon=|f(c)|$. By continuity, there exists $\delta>0$ such that for all $x \in[a, b]$, if $|x-c|<\delta$, then $|f(x)-f(c)|<|f(c)|$. This implies
$$
-f(c)<f(x)-f(c)<f(c) .
$$

Adding $f(c)$ to all parts of the inequality yields $0<f(x)$ for $x$ in the interval $(c-\delta, c) \cap[a, b]$. Similarly, subtracting $f(c)$ from all parts of the inequality yields $f(x)<0$ for $x$ in the interval $(c, c+\delta) \cap[a, b]$. Now, consider the set of rational numbers in $[a, b]$. Since $\mathbb{Q}$ is dense in $\mathbb{R}$, there exist rational numbers in both $(c-\delta, c) \cap[a, b]$ and $(c, c+\delta) \cap[a, b]$. However, we have shown that for any rational $x$ in these intervals, $f(x)$ has opposite signs, which is a contradiction.

\textbf{You are almost there. Instead of saying there exists a rational number in the intervals you should say there exists an irrational number, say $x_I$, which we know $f(x_I)=0.$ From then you can continue with your contradiction argument. You also need to note that for any number in those intervals $f(x)$ has opposite signs.}

Therefore, our assumption that there exists $c$ such that $f(c) \neq 0$ is false, and we conclude that $f(x)=0$ for all $x \in[a, b]$.

\end{document}
