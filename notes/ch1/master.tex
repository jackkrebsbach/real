
\documentclass{report}

\input{preamble}
\input{macros}
\input{letterfonts}

\title{The Real Numbers}
\author{Jack Krebsbach }

\date{Sep 4}

\begin{document}
\maketitle

\section{The Real Numbers}

\subsection{The irrationality of the square root of 2 }

\subsection{Preliminaries}

\texttt{Notation}
\begin{multicols}{2}
\begin{itemize}
  \item $ \rightarrow$ For all/each/every
  \item $\exists \rightarrow$ There exists
  \item $\mathbb{R} \setminus \mathbb{Q} \rightarrow$ Irrationals
  \item $\mathbb{R} \rightarrow$ Real numbers
  \item $\mathbb{Z} \rightarrow$ Integers
  \item $\mathbb{Q} \rightarrow$ Rational numbers
  \item $\mathbb{N} \rightarrow$ Natural numbers
  \item $BWOC \rightarrow$ By way of contradiction
  \item $\rightarrow\!\leftarrow \rightarrow$ Contradiction
  \item $! \rightarrow$ Unique/factorial
  \item $\qed \rightarrow$ End of proof (Quod Erat Demonstrandum)
  \item $\epsilon \rightarrow$ Epsilon, usually a small positive quantity
  \item $\ni \rightarrow$ Such that
\end{itemize}
\end{multicols}

\thm{}{
 Two real numbers $a$ and $b$ are equal if and only if for every real number $\epsilon > 0$ it follows that $|a-b| < \epsilon.$ 
}

\begin{myproof}

  \par
  $\Rightarrow$
  If $a=b$ then we have $|a-b| =0$. No matter which  $\epsilon >  0 \in \mathbb{R}$ is chosen we have that  $|a-b|=0 < \epsilon.$ Thus, $a=b$  \par
$\Leftarrow$
Suppose, by way of contradiction, that $a\neq b$. We know that $\forall \epsilon > 0$ and $|a-b| < \epsilon$ . Let $\epsilon_{0} = \frac{|a-b|}{2}$, then it is clear that $|a-b| < \frac{|a-b|}{2} = \epsilon_{0}$ is false. Thus, with this contradiction we overturn our assumption and conclude a must equal b.
\end{myproof}

\subsection{The Axiom of Completeness}

\dfn{Bounded Above}{
A set $A \subseteq \mathbf{R}$ is bounded above if there exists a number $b \in \mathbf{R}$ such that $a \leq b$ for all $a \in A$. The number $b$ is called an upper bound for $A$.
Similarly, the set $A$ is bounded below if there exists a lower bound $l \in \mathbf{R}$ satisfying $l \leq a$ for every $a \in A$.
}
\bigskip

\dfn{The Supremum of a Set}{
A real number $s$ is the \textit{Supremum} or the least upper bound for a set $A \subseteq \mathbf{R}$ if: \par
(i) $s$ is an upper bound for $A$;\par
(ii) if $b$ is any upper bound for $A$, then $s \leq b$.
}

\thm{\textbf{Lemma 1.3.8}}{ Assume $s \in \mathbf{R}$ is an upper bound for a set $A \subseteq \mathbf{R}$. Then, $s=\sup A$ if and only if, for every choice of $\epsilon>0$, there exists an element $a \in A$ satisfying $s-\epsilon<a$.
}

\begin{myproof}
\par
$\Rightarrow$
Let  $s = \sup A$ and arbitrarily choose any $\eps >0$. Suppose, by way of contradiction, there does not exist $a \in A$ with $a > s- \epsilon$. So for all $a \in A$ we have that $ a \leq s - \eps < s$. This means that $s-\epsilon$ is an upper bound  of $A$. However, $s$ is  the supremum of or the least upper bound  of $A$ , so this is a contradiction. Thus, there must exist an element $a \in A$ such that $s- \epsilon <a.$

\par
$\Leftarrow$
Let $ \eps > 0$ and suppose  there exists $a \in A$ with $s-\epsilon < a$ and we know that $s$ is an upper bound  of $A$. To show that $s$ is the least upper bound, by way of contradiction, suppose that $b<s$ and $b$ is another upper bound of $A$.
\par 
Consider $\epsilon_{0} = s - b > 0.$ By hypothesis there exists $a$ with $s - \epsilon_{0} < a \implies s - (s - b) < a \implies  b < a.$ As we know that $b$ is an upper bound of $A$ this is impossible. Therefore it must be that $b \geq s$. Hence, $s \leq b$ and $s = \sup A.$

\end{myproof}
\bigskip

\subsection{Consequences of Completeness}
\thm{
The Archimedes Property 
}{
  (a) For all $x \in \RR$ there exists $n \in \NN$ such that $n > x$.

  (b) For all $y \in \RR$ with $y \neq 0$, there exists $n \in \NN$ such that $\frac{1}{n} < y.$
}

\begin{myproof}
  
  (a) Suppose, by way of contradiction, that the natural numbers are bounded. Let $\alpha \in \RR$ be an upper bound, so $n \leq \alpha$ for all $n \in \NN$. Let $\beta = \sup \NN$ [exists by completeness]. Now $\beta - 1$ is not an upper bound. By our theorem, there exists $n_0 \in \NN$ such that $\beta - 1 < n_0.$ \par
  So $\beta < n_0 +1 \in \NN$. This is a contradiction because we assumed that $\beta$ was the supremum of the natural numbers. Thus, $\NN$ is unbounded. For any $\alpha \in \RR$, we can find a natural number that is larger than $\alpha.$

  (b) To show why be is the case you can consider $x = \frac{1}{y}$ where $y \neq =0.$
\end{myproof}

\subsection{Examples}

\begin{enumerate}
    
\item
Show that $\bigcap^{\infty}_{n=1} (0, \frac{1}{n}) = \emptyset.$

\begin{myproof}
    Suppose otherwise, that the intersection is no the empty set, and let $x \in \bigcap^{\infty}_{n=1}.$ Then it follows $x \in (0, \frac{1}{n})$ for all $n \in \NN$. By the corollary of the Archimedes Property, there exists  $n_\star \in \NN$ with $\frac{1}{n_\star} < x.$ Then there does not exist $x$ such that $x \in \bigcap^{\infty}_{n=1} (0,\frac{1}{n})$. Therefore, $\bigcap^{\infty}_{n=1} (0, \frac{1}{n}) = \emptyset.$
\end{myproof}

\item Show there does not exist a smallest positive number
  \begin{myproof}
      To show that there does not exist a smallest positive number, suppose otherwise. Let $x \in \RR^+.$ By Archimedes property there exists $n_0 \in \NN$ such that $\frac{1}{n_0} < x.$ So $x$ cannot be the smallest.
  \end{myproof}
\end{enumerate}
\pagebreak

\thm{The Nested Cells Property}{
For each $n \in \NN$, assume we are given a closed interval $I_n=\left[a_n, b_n\right]=\left\{x \in \RR: a_n \leq x \leq b_n\right\}$. Assume also that each $I_n$ contains $I_{n+1}$. Then, the resulting nested sequence of closed intervals
$$
I_1 \supseteq I_2 \supseteq I_3 \supseteq I_4 \supseteq \cdots
$$
has a non-empty intersection; that is, $\bigcap_{n=1}^{\infty} I_n \neq \emptyset$.
}

\begin{myproof}
  Let $$A=\{a_1, a_2,a_3, \dots\}$$  and $$B=\{b_1,b_2,b_3,
  \dots\}.$$ $A$ is bounded above by one element of $B$ and $B$ is bounded below by any $a \in A$. 
  Let $x = \sup A$ which implies that $a_n \leq x$ for all $n \in \NN$. Since $b$ is an upper bound of $A$ and $x = \sup A$, we have that $ x \leq b_n$ for all $n$ and that $a_n \leq x \leq b_n$ for all $n$.Thus, $x \in [a_n, b_n]$ for all $n$ and $ x \in \bigcap_{n=1}^\infty I_n.$ 
\end{myproof}


\subsection{Intersection Examples}
\begin{itemize}
  \item $\bigcap (0, \frac{1}{n}) = \emptyset$

  \item $\bigcap [0, \frac{1}{n}] = \{0\}$
  \item $\bigcap (0, \frac{1}{n}] = \emptyset$
  \item $\bigcap (-\frac{1}{n}, \frac{1}{n}) = \{0\}$
\end{itemize}

\thm{Density of $\QQ$ in $\RR$}{
  Let $a,b \in \RR$. Then there exist $n \in \QQ$ with $a < r < b$.
}

\begin{myproof}
  Since $a,b \in \RR$, without loss of generality let $a <b.$ Now, we have that $b-a >0.$ By the corollary to the Archimedes principle there exist $n_\star \in \NN$ with  $\frac{1}{n_\star} < b-a.$ \par
  Consider $n_\star a \in \RR.$ Pick $m \in \NN$ so that $$ m - 1 \leq n_{\star} a < m.$$

In other words, we choose the smallest of natural numbers greater than $n_\star a.$ By chance, it may be that one less than that number is $n_\star a.$ so we end up with the equality $m-1 \leq n_\star a.$ \par

We have that $$ n_\star a < m \implies a < \frac{m}{n_\star}$$ and, $$ m \leq n_\star a + 1,$$ because we know that $$
m - 1 \leq n_{\star} a.$$

Next, $$\frac{1}{n_\star}< (b-a) \implies 1 < n_\star (b -a) \implies 1 < n_\star b - n_\star a \implies n_\star a < n_\star b - 1 \implies a < \frac{n_\star b - 1}{n_\star} \implies a < b - \frac{1}{n_\star}.$$
We take $$ m \leq n_\star a + 1 <  n_\star[b - \frac{1}{n_\star}] + 1 = n_\star b - 1 + 1 = n_\star b$$


So, $m < n_\star b$ which means $ \frac{m}{n_\star} < b.$

\end{myproof}


\subsection{Cardinality}

\dfn{1-1 and onto}{
  \begin{itemize}
    \item $f: a \rightarrow b $ is 1-1 if $a_1 \neq a_2$ implies that $f(a_1) \neq f(a_2).$
    \item $f: a \rightarrow b$ is onto if for every $b \in B$ there exists $a \in A $ such that $f(a) =b.$
      
  \end{itemize}
}

\dfn{Cardinality}{
  A set $A$ has the same cardinality as $B$ if there exists a 1-1 and onto function, $f: A \rightarrow B$. We have $|A| = |B|$ or $A ~ B.$
}

\subsubsection{Examples}
\begin{enumerate}
  \item Show that $\NN \sim \ZZ.$
    \begin{myproof}
        
\[ 
 f(n) = 
  \begin{cases} 
  odd & \frac{n-1}{2} \\
  even & -\frac{n}{2}
   \end{cases}
\]
    \end{myproof}

  \item Show that $[0,1] \sim [\pi,5]$.
    \begin{myproof}
        $$y = (5- \pi)x + \pi$$
    \end{myproof}

    Is $(0,1] \sim (\pi,5)?$ Yes but we need to be careful how we show.
\end{enumerate}


\dfn{Infinite}{
  \begin{itemize}
  \item A set is \textit{finite} if $|A| = |\NN_n| n \in \NN$
  \item A set is \textit{infinite} if it is not \textit{finite}.
  \end{itemize}
}
\dfn{Countable}{
  \begin{itemize}
  \item An infinite set is \textit{countable} if $|A| = |\NN|$.
  \item An infinite set is \textit{uncountable} if it is not \textit{countable}.
  \end{itemize}
}


\subsubsection{Examples}

\begin{enumerate}
  \item $\QQ$?
  \item $\RR \setminus \QQ$?
  \item $\RR \setminus \QQ$?
  \item Is the union of countable sets countable?
\end{enumerate}

\pagebreak
  \thm{}{
    Let $|A|=n$ and $|B|=m.$ Then $A \cup B$ is finite.
  }

  \begin{myproof}
    Let
    $$A = \{a_1, a_2, \dots, a_n\}$$ and
    $$B = \{b_1, b_2, \dots, b_m\}.$$

    Define $n_1=\min \{k \colon k \in \NN, b_k \not \in A\}$ and $n_2 = \min \{k \colon b_k \not \in A, k > n_1\}$. Generally, $n_j = \min \{k : b_k \in A, k > n_{j-1} \}.$ Then $|A|=n$ and $|B \setminus A|=j.$

    $$
f(l) = 
  \begin{cases} 
  a_l &  l \leq n \\
  b_l & n+1 \leq l \leq n+j
   \end{cases}
    $$
  \end{myproof}


  \thm{}{
    The subset, $A$,  of a finite set $B$ is finite. That is $A \subset B$ is finite if $B$ is finite.l
  }
  \begin{myproof}
    Let $|B|=n$, $B=\{b_1, b_2,\dots,b_n\}.$

    Then let $$n_1 = \min \{k \colon b_k \in A\},$$
    $$n_2 = \min \{k \colon b_k \in A, k > n_1\},$$ and $$n_j = \min \{k \colon b_k \in A, k > n_{j-1}\}.$$ Define the bijections $f \colon \NN_{n_j} \rightarrow A$ where $f(j) = b_{n_j}$ is a finite bijection.
  \end{myproof}


  \thm{}{
    Let $A$ and $B$ be sets with $A \subset B.$
    \begin{enumerate}
      \item If $B$ is countable, then $A$ is countable or finite.
      \item If $A$ is uncountable, then $B$ is uncountable.
    \end{enumerate}
  }

  \begin{myproof}
      \begin{enumerate}

        \item Since $B$ is countable let $B = \{b_j \colon j \in\NN$. To "count" A let $n_1 = \min \{k \colon b_k \in A\}$ and $n_2 = \min \{k \colon b_k \in A, k > n_1\}$. Generally, $$n_j = \min \{k \colon b_k \in A, k > n_{j-1}\}.$$

    If there does not exist $k \in b_k \in A$ then $|A| =m,$ which is finite. We have a function $f(m)= b_{n_m}$ which is onto A in a 1-1 manner. 
  \item Contrapositive of 1!
      \end{enumerate}
  \end{myproof}

  \thm{}{
    The countable  union of countable sets is countable.
  }
  \begin{myproof}
    Let $A_n$ where $n \in \NN$ be a collection of countable sets. So that  $A_n = \{A_1, A_2, \dots \}.$ We can list off elements in each set within the collection $A_n = \{a_{nm} \colon m \in \NN\}.$ We want to show that $\bigcup_{n=1}^{\infty}\{a_{nm} \colon m \in \NN\}$ is countable. Consider $f(a_{nm}) = 2^n3^m.$ This is 1-1 by prime factorization, known as the Fundamental Theorem of Arithmetic. The set $\{2^n3^m \colon n,m \in \NN \} \subset \NN$ and is countable.
  \end{myproof}

  \thm{}{
    $\QQ$ is countable.
  }

  \begin{myproof}
    Consider $A_n = \{ \frac{m}{n} \colon m \in \NN \}.$ Then $
    \bigcup_{n=1}^{\infty} A_n$ is countable. The set $B_n = \{ -\frac{m}{n} \colon m \in \NN \}$ is also countable. The set $\{0\}$ is finite and thus countable. Altogether we have $$ \QQ = \bigcup_{n=1}^{\infty} A_n \cup \bigcup_{n=1}^{\infty} B_n \cup \{0\}$$ is countable by the previous theorem.
  \end{myproof}

  \subsection{Cantors Diagonilization}


\thm{}{
   $\RR$ is uncountable. The open interval $(0,1)=\{x \in \mathbf{R}: 0<x<1\}$ is uncountable.
}
  
\begin{myproof}

  Suppose that the interval $(0,1)$ is countable. Then there exists a bijection $f \colon \NN \rightarrow (0,1).$ We can express this like so.

  \bigskip

 \begin{centering}
     
\begin{tabular}{c c c c c c c c c c c}
 $\mathbf{N}$ & & $(0,1)$ & & & & & & & & \\
\hline 1 & $\longleftrightarrow$ & $f(1)$ & $=$ & $a_{11}$ & $a_{12}$ & $a_{13}$ & $a_{14}$ & $a_{15}$ & $a_{16}$ & $\cdots$ \\
 2 & $\longleftrightarrow$ & $f(2)$ & $=$ &.$a_{21}$ & $a_{22}$ & $a_{23}$ & $a_{24}$ & $a_{25}$ & $a_{26}$ & .. \\
 3 & $\longleftrightarrow$ & $f(3)$ & $=$ &.$a_{31}$ & $a_{32}$ & $a_{33}$ & $a_{34}$ & $a_{35}$ & $a_{36}$ & .. \\
 4 & $\longleftrightarrow$ & $f(4)$ & $=$ &.$a_{41}$ & $a_{42}$ & $a_{43}$ & $a_{44}$ & $a_{45}$ & $a_{46}$ & $\ldots$ \\
 5 & $\longleftrightarrow$ & $f(5)$ & $=$ &.$a_{51}$ & $a_{52}$ & $a_{53}$ & $a_{54}$ & $a_{55}$ & $a_{56}$ & . \\
 6 & $\longleftrightarrow$ & $f(6)$ & $=$ & $a_{61}$ & $a_{62}$ & $a_{63}$ & $a_{64}$ & $a_{65}$ & $a_{66}$ & $\cdots$ \\
$\vdots$ & & & & $\vdots$ & $\vdots$ & $\vdots$ & $\vdots$ & $\vdots$ & : & \\

\end{tabular}
\end{centering}   

Where $a_{nm} \in \{0,\dots,9\}.$ We can define $y=y_1y_2\dots$ where
$$ y_{mm} = 
  \begin{cases} 
    2  &  a_{mm} \geq 5 \\
    7 &   a_{mm} \leq 4
   \end{cases}
$$
Thus, we have constructed $y$ that differs from every element in our list. This says that $(0,1) \subset \RR$ is uncountable. Thus, $\RR$ is uncountable.

\end{myproof}

Second proof

\begin{myproof}
  To show that $\RR$ is uncountable, suppose otherwise. Then $\RR$ can be written $\RR = \{x_1, x_2, x_3,\dots\}.$ We create a $1-1$ correspondance with the natural numnbers. Pick $x_1$ and choose a close bounded interval $I_1$ so that $x_1 \not\in I_1.$ Choose $I_2$ and $x_2$ so that $I_2 \subset I_1$ and $x_2 \not \in  I_2.$ In similar fashion create $I_3 \subset I_2$ and choose $x_3 \not\in I_3.$  We are creating a sequence of nested, close bounded intervals with $x_n \not \in I_n.$ \par
  By the nested inerval property $\bigcap_{n=1}^{\infty} I_n \neq \emptyset.$ So there exists $\alpha \in \bigcap_{n=1}^{\infty} I_n.$ Since $\alpha$ is a real number there exists $n_0 \in \NN$ where $x_{n_0}=\alpha.$ But we know $x_{n_0} \not \in I_n$ by construction, and have found a contradiction. So $\RR$ is uncountable.
\end{myproof}

\subsection{The Power Set of the Set}

\dfn{Power Set}{
Let $A$ be a set. Then the power set of $A$, $\mathcal{P}(A)$ is the set of subsets  of $A.$
}

Examples: $A={1,2}$. Then $\mathcal{P}(A) =\{\{1\},\{2\},\{1,2\}, \emptyset\}$

$$|\mathcal{P}(A)| = 2^{|A|}$$

 \end{document}


