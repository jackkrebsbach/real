
\documentclass{report}

\input{preamble}
\input{macros}
\input{letterfonts}

\title{The Real Numbers}
\author{Jack Krebsbach }

\date{Sep 4}

\begin{document}
\maketitle

\section{The Real Numbers}

\subsection{The irrationality of the square root of 2 }

\subsection{Preliminaries}

\texttt{Notation}
\begin{multicols}{2}
\begin{itemize}
  \item $\forall \rightarrow$ For all/each/every
  \item $\exists \rightarrow$ There exists
  \item $\mathbb{R} \setminus \mathbb{Q} \rightarrow$ Irrationals
  \item $\mathbb{R} \rightarrow$ Real numbers
  \item $\mathbb{Z} \rightarrow$ Integers
  \item $\mathbb{Q} \rightarrow$ Rational numbers
  \item $\mathbb{N} \rightarrow$ Natural numbers
  \item $BWOC \rightarrow$ By way of contradiction
  \item $\rightarrow\!\leftarrow \rightarrow$ Contradiction
  \item $! \rightarrow$ Unique/factorial
  \item $\qed \rightarrow$ End of proof (Quod Erat Demonstrandum)
  \item $\epsilon \rightarrow$ Epsilon, usually a small positive quantity
  \item $\ni \rightarrow$ Such that
\end{itemize}
\end{multicols}

 Two real numbers $a$ and $b$ are equal if and only if for every real number $\epsilon > 0$ it follows that $|a-b| < \epsilon.$ 

\begin{proof}

  \par
  $\rightarrow$
  If $a=b$ then we have $|a-b| =0$. No matter which  $\epsilon >  0 \in \mathbb{R}$ is chosen we have that  $|a-b|=0 < \epsilon.$ Thus, $a=b$  \par
$\leftarrow$
Suppose, by way of contradiction, that $a\neq b$ and $\forall \epsilon > 0$, we have $|a-b| < \epsilon$ . Let $\epsilon_{0} = \frac{|a-b|}{2}$, then it is clear that $|a-b| < \frac{|a-b|}{2} = \epsilon_{0}$ is false. Thus, with this contradiction we overturn our assumption and conclude a must equal b.
\end{proof}

\subsection{The Axiom of Completeness}

\dfn{Bounded Above}{
A set $A \subseteq \mathbf{R}$ is bounded above if there exists a number $b \in \mathbf{R}$ such that $a \leq b$ for all $a \in A$. The number $b$ is called an upper bound for $A$.
Similarly, the set $A$ is bounded below if there exists a lower bound $l \in \mathbf{R}$ satisfying $l \leq a$ for every $a \in A$.
}
\bigskip

\dfn{Supremum of a Set}{
A real number $s$ is the \textit{Supremum} or the least upper bound for a set $A \subseteq \mathbf{R}$ if: \par
(i) $s$ is an upper bound for $A$;\par
(ii) if $b$ is any upper bound for $A$, then $s \leq b$.
}

\textbf{Lemma 1.3.8}. Assume $s \in \mathbf{R}$ is an upper bound for a set $A \subseteq \mathbf{R}$. Then, $s=\sup A$ if and only if, for every choice of $\epsilon>0$, there exists an element $a \in A$ satisfying $s-\epsilon<a$.

\begin{proof}
\par
$\lthen$
Let  $s = \sup A$ and arbitralily choose any $\eps >0$. Suppose, by way of contradiction, there does not exist $a \in A$ with $a > s- \epsilon$. So for all $a \in A$ we have that $ a \leq s - \eps < s$. This means that $s-\epsilon$ is an upperbound of $A$. Thus, there must exist an element $a \in A$ such that $s- \epsilon <a.$


\par
$\leftarrow$
Let $ \eps > 0$ and suppose there exists $a \in A$ with $s-\epsilon < a$ and we know that $s$ is an upperbound of $A$. To show that $s$ is the least upperbound, by way of contradiction, suppose that $b<s$ and $b$ is another upperbound of $A$.
\par 
Consider $\epsilon_{0} = s - b > 0.$ By hypothesis there exists $a$ with $s - \epsilon_{0} < a \implies s - (s - b) < a \implies  b < a.$ As we know that $b$ is an upperbound of $A$ this is impossible. Therefore it must be that $b \geq s$. Hence, $s \leq b$ and $s = \sup A.$
\end{proof}

\end{document}

