
\documentclass{report}

\input{preamble}
\input{macros}
\input{letterfonts}

\title{Sequences and Series}
\author{Jack Krebsbach }

\date{Oct 4}

\begin{document}
\maketitle

\section{Sequences and Series}

\dfn{Sequence}{
  A sequence is a function from $f \colon \NN \rightarrow \RR.$

  Examples:
  \begin{enumerate}
    \item $(a_n)$ 
    \item ($a_1, a_2, a_3, \dots, a_n$)
  \end{enumerate}

}

\dfn{Convergence}{
  A sequence, $(a_n)$, converges to a point, $x,$ if for all $\epsilon >0$ there exist $N \in \NN$  such that for all $n>N$, $|a_n - x| < \epsilon.$
}

\thm{Uniqueness of Limits.}{
The limit of a sequence, when it exists, must me unique.
}

\begin{myproof}
  Let $(x_n)$ be a convergent series that converges to $x.$ By way of contradiction, suppose that $(x_n) \rightarrow y$ where $x \neq y$ and $x < y$. Let $\epsilon = \frac{1}{3}(y-x).$  Since $(x_n)$ converges to $x$ there exists $N_x \in \NN$ such that for all $n> N_x$, $|x_n - x| < \epsilon.$ Similarly, since $(x_n)$ converges to $y$ there exists $N_Y \in \NN$ such that for all $n> N_y$, $|x_n - y| <\epsilon.$ 

  \par Let $N = \max\{N_x, N_y\}.$  Then $x_{N+2} \in \mathcal{B}(x,\epsilon) \cap \mathcal{B}(y,\epsilon)$. This is a contradiction, $x_{N+2} \not\in \mathcal{B}(x,\epsilon) \cap \mathcal{B}(y,\epsilon).$ Thus,  $x=y$ and limits are unique!
\end{myproof}

\pagebreak




















 \end{document}


